%%%%%%%%%%%%%%%%%%%%%%%%%%%%%%%%%%%%%%%%%%%%%%%%%%%%%%%%%%%%%%%%%%%%%%%%%%%%%%%%%%%%%%%%%%%%%%%%%%
% Modelo de TCC do Bacharelado em Engenharia de Computação da UNIFESP (por Profa. Fernanda Rossi)
% Versão 1.0 (26/novembro/2025)
%%%%%%%%%%%%%%%%%%%%%%%%%%%%%%%%%%%%%%%%%%%%%%%%%
%%%%%%%%%%%%%%%%%%%%%%%%%%%%%%%%%%%%%%%%%%%%%%%%%

% Baseado no Modelo de Documentos Academicos do ABNTex2  

\documentclass[	12pt, openright, twoside, a4paper, english, brazil]{abntex2}

% ---
% Pacotes fundamentais 
% ---
\usepackage{cmap}				% Mapear caracteres especiais no PDF
%\usepackage{lmodern}			% Usa a fonte Latin Modern			
\usepackage{times}
\usepackage[T1]{fontenc}			% Selecao de codigos de fonte.
\usepackage[utf8]{inputenc}		% Codificacao do documento (conversão automática dos acentos)
\usepackage{hyperref}			% Deve vir antes de lastpage
% \usepackage{lastpage}			% Usado pela Ficha catalográfica
%\usepackage{natbib}
\usepackage{indentfirst}			% Indenta o primeiro parágrafo de cada seção.
\usepackage{color}				% Controle das cores
\usepackage{graphicx}			% Inclusão de gráficos
% ---

% ---
% Pacotes de citações
% ---
\usepackage[brazilian,hyperpageref]{backref}	 % Paginas com as citações na bibl
\usepackage[alf]{abntex2cite}	% Citações padrão ABNT

% Define comando htmladdnormallink para compatibilidade com bibliografia
\providecommand{\htmladdnormallink}[2]{\href{#2}{#1}}

% --- 
% CONFIGURAÇÕES DE PACOTES
% --- 

% ---
% Configurações do pacote backref
% Usado sem a opção hyperpageref de backref
\renewcommand{\backrefpagesname}{Citado na(s) página(s):~}
% Texto padrão antes do número das páginas
\renewcommand{\backref}{}
% Define os textos da citação
\renewcommand*{\backrefalt}[4]{
	\ifcase #1 %
		Nenhuma citação no texto.%
	\or
		Citado na página #2.%
	\else
		Citado #1 vezes nas páginas #2.%
	\fi}%
% ---

% numeração de figuras e tabelas 
\counterwithout{figure}{section}
\counterwithout{table}{section}

%\renewcommand\tablename{Tabela{\arabic{chapter}.}}


% ---
% Informações de dados para CAPA e FOLHA DE ROSTO
% ---
\titulo{Sistema de Agricultura de Precisão para o Manejo Sustentável de Recursos Hídricos baseado na Fusão de Dados de Sensores IoT e Imagens de Satélite}
\autor{Fernando Daniel Marcelino}
\local{São José dos Campos, SP}
\data{Dezembro de 2025}
\orientador{Profa. Dra. Fernanda Quelho Rossi}
%\coorientador{Profa. Dra. Nome da Coorientadora}
\instituicao{%
  Universidade Federal de São Paulo -- UNIFESP
  \par
  Instituto de Ciência de Tecnologia
  \par
  Bacharelado em Engenharia de Computação}
\tipotrabalho{Trabalho de Graduação}
% O preambulo deve conter o tipo do trabalho, o objetivo, 
% o nome da instituição e a área de concentração 
\preambulo{Trabalho de conclusão de curso apresentado ao Instituto de Ciência e Tecnologia – UNIFESP, como parte das atividades para obtenção do título de Bacharel em Engenharia de Computação.}
% ---

% informações do PDF
\makeatletter
\hypersetup{
     	%pagebackref=true,
		pdftitle={\@title}, 
		pdfauthor={\@author},
    	pdfsubject={\imprimirpreambulo},
	    pdfcreator={LaTeX with abnTeX2},
		pdfkeywords={abnt}{latex}{abntex}{abntex2}{trabalho acadêmico}, 
		colorlinks=true,       		% false: boxed links; true: colored links
    	linkcolor=blue,          	% color of internal links
    	citecolor=blue,        		% color of links to bibliography
    	filecolor=magenta,      	% color of file links
		urlcolor=blue,
		bookmarksdepth=4
}
\makeatother
% --- 
% --- 
% Espaçamentos entre linhas e parágrafos 
% --- 
% O tamanho do parágrafo é dado por:
\setlength{\parindent}{1.25cm}
% Controle do espaçamento entre um parágrafo e outro:
\setlength{\parskip}{0.2cm}  % tente também \onelineskip
% ---

% compila o indice
% ---
\makeindex
% ---

% ----
% Início do documento
% ----
\begin{document}
% Retira espaço extra obsoleto entre as frases.
\frenchspacing 

% ----------------------------------------------------------
% ELEMENTOS PRÉ-TEXTUAIS
% ----------------------------------------------------------
% \pretextual

% ---
% Capa
% ---
\begin{capa}
  \begin{center}
   \includegraphics[width=.25\textwidth]{logo-unifesp.pdf}
    \vspace*{\fill}
    
    {\ABNTEXchapterfont\large\imprimirautor}
    \vspace*{\fill}
    
    {\ABNTEXchapterfont\bfseries\Large\imprimirtitulo}
    \vspace*{\fill}\vspace*{\fill}
    
   \imprimirlocal
   \end{center}
\end{capa}

% ---
% Folha de rosto
% (o * indica que haverá a ficha bibliográfica)
% ---
\imprimirfolhaderosto*
% ---
% ---
% Inserir folha de aprovação
% ---
% Isto é um exemplo de Folha de aprovação, elemento obrigatório da NBR
% 14724/2011 (seção 4.2.1.3). Você pode utilizar este modelo até a aprovação
% do trabalho. Após isso, substitua todo o conteúdo deste arquivo por uma
% imagem da página assinada pela banca com o comando abaixo:
%
% \includepdf{folhadeaprovacao_final.pdf}
%
% \begin{folhadeaprovacao}
%   \begin{center}
%     {\ABNTEXchapterfont\large\imprimirautor}

%     \vspace*{\fill}\vspace*{\fill}
%     {\ABNTEXchapterfont\bfseries\Large\imprimirtitulo}
%     \vspace*{\fill}
    
%     \hspace{.45\textwidth}
%     \begin{minipage}{.5\textwidth}
%         \imprimirpreambulo
%     \end{minipage}%
%     \vspace*{\fill}
%    \end{center}
    
%    Trabalho aprovado em 10 de Dezembro de 2025:

%    \assinatura{\textbf{\imprimirorientador} \\ Orientador} 
%    \assinatura{\textbf{Profa. Dra. Nome da Professora} \\ Convidado 1}
%    \assinatura{\textbf{Prof. Dr. Nome do Professor} \\ Convidado 2}
%    %\assinatura{\textbf{Professor} \\ Convidado 3}
      
%    \begin{center}
%     \vspace*{0.5cm}
%     {\large\imprimirlocal}
%     \par
%     {\large\imprimirdata}
%     \vspace*{1cm}
%   \end{center}
  
% \end{folhadeaprovacao}
% ---

% ---
% Dedicatória
% ---
\begin{dedicatoria}
   \vspace*{\fill}
   \centering
   \noindent
   \textit{ Este trabalho é dedicado aos meus pais que sempre me apoiaram e me incentivaram. } \vspace*{\fill}
\end{dedicatoria}
% ---

% ---
% Agradecimentos
% ---
\begin{agradecimentos}
Agradeço à minha família e em especial aos meus pais, Edmilson Aparecido Marcelino e Lucineia Domingues Marcelino, que dedicaram seus esforços para a minha educação.

Agradeço à minha orientadora, Profa. Dra. Fernanda Quelho Rossi, pelo apoio, paciência e ensinamentos ao longo deste trabalho. Não apenas durante a elaboração desta monografia, mas durante toda a minha trajetória acadêmica.

Agradeço também aos meus amigos e colegas de curso, que tornaram essa jornada mais leve e divertida. Levo cada momento compartilhado com vocês como aprendizado e crescimento pessoal.

Agradeço à Universidade Federal de São Paulo (UNIFESP) e ao Instituto de Ciência e Tecnologia (ICT) por me proporcionar todo amparo e estrutura necessária para a minha formação acadêmica.

Por fim, agradeço a todos que, de alguma forma, contribuíram para a realização deste trabalho.

\end{agradecimentos}
% ---

% ---
% Epígrafe
% ---
\begin{epigrafe}
    \vspace*{\fill}
	\begin{flushright}
		\textit{``A educação é a arma mais poderosa \\
		que você pode usar para mudar o mundo.'' \\
		(Nelson Mandela)}
	\end{flushright}
\end{epigrafe}
% ---

% ---
% RESUMOS
% ---
\begin{resumo}
A escassez e a distribuição desigual dos recursos hídricos colocam a agricultura sob forte pressão para produzir 
mais, com menor desperdício de água e menor impacto ambiental. Diante desse cenário, e em consonância com os Objetivos de 
Desenvolvimento Sustentável (ODS 2, 6 e 12), esta monografia descreve o projeto de um sistema de agricultura de precisão 
para o manejo sustentável de recursos hídricos, fundamentado na integração de duas fontes principais de informação: medições de redes 
de sensores em campo e dados de sensoriamento remoto orbitais. O trabalho aborda, inicialmente, os fundamentos teóricos de Internet 
das Coisas (IoT), sensoriamento remoto e fusão de dados, com ênfase em sensores de umidade do solo, índices espectrais derivados de imagens 
multiespectrais e produtos de missões espaciais como Sentinel e Landsat. Em seguida, é proposta uma arquitetura em camadas 
que abrange aquisição, transmissão, armazenamento e tratamento de dados, incluindo rotinas de pré-processamento e sincronização 
temporal. A partir desse fluxo, são definidos conjuntos de atributos que combinam variáveis de solo, clima e informações espectrais para 
alimentar modelos de aprendizado de máquina voltados à estimativa de indicadores de estado hídrico e à geração de mapas de recomendação 
de irrigação em nível de talhão. Complementarmente, delineia-se a implementação de um protótipo de interface visual para apoio à decisão, 
capaz de apresentar mapas, séries temporais e alertas de forma intuitiva. Como contribuição, o trabalho organiza boas práticas de projeto 
de dados, documenta uma metodologia de fusão IoT–satélite e oferece uma base estruturada para desenvolvimentos futuros em irrigação inteligente 
e agricultura de precisão sustentável.
 
\vspace{\onelineskip}
\noindent
\textbf{Palavras-chave}: agricultura de precisão; recursos hídricos; irrigação inteligente; Internet das Coisas; sensoriamento remoto.
\end{resumo}


\begin{resumo}[Abstract]
\begin{otherlanguage*}{english}
Water scarcity and the uneven distribution of water resources place agriculture under strong pressure to 
increase production while reducing water waste and environmental impacts. In this context, and in accordance 
with the Sustainable Development Goals (SDGs 2, 6 and 12), this monograph describes the design of a precision agriculture 
system for sustainable water resources management, grounded in the integration of two main sources of 
information: measurements from in-field IoT sensor networks and orbital remote sensing data. The work first discusses the theoretical 
foundations of the Internet of Things, remote sensing and data fusion, with textitasis on soil moisture sensors, spectral 
indices derived from multispectral imagery and products from space missions such as Sentinel and Landsat. Next, a layered 
architecture is proposed, covering data acquisition, transmission, storage and processing, including preprocessing routines 
and temporal synchronization. Based on this flow, feature sets are defined that combine soil, climate and spectral variables to feed 
machine learning models aimed at estimating indicators of crop water status and generating irrigation recommendation maps at field level. 
Additionally, the implementation of a visual decision-support prototype is outlined, capable of presenting maps, time series and alerts in an 
intuitive way. As its main contributions, this work organizes data-project best practices, documents a methodology for IoT–satellite 
data fusion and provides a structured basis for future developments in smart irrigation and sustainable precision agriculture.
 
\vspace{\onelineskip}
\noindent 
\textbf{Key-words}: precision agriculture; water resources; smart irrigation; Internet of Things; remote sensing.
\end{otherlanguage*}
\end{resumo}


% ---
% inserir lista de ilustrações
% ---
\pdfbookmark[0]{\listfigurename}{lof}
\listoffigures*
\cleardoublepage
% ---

% ---
% inserir lista de tabelas
% ---
\pdfbookmark[0]{\listtablename}{lot}
\listoftables*
\cleardoublepage
% ---

% ---
% inserir lista de abreviaturas e siglas
% ---
\begin{siglas}
  \item[ABNT] Associação Brasileira de Normas Técnicas
  \item[ADC] Conversor Analógico--Digital (\textit{Analog-to-Digital Converter})
  \item[ALOS] \textit{Advanced Land Observing Satellite}
  \item[API] Interface de Programação de Aplicações (\textit{Application Programming Interface})
  \item[ASTER] \textit{Advanced Spaceborne Thermal Emission and Reflection Radiometer}
  \item[CLI] Interface de Linha de Comando (\textit{Command-Line Interface})
  \item[COG] \textit{Cloud Optimized GeoTIFF} (GeoTIFF otimizado para acesso em nuvem)
  \item[CPU] Unidade Central de Processamento (\textit{Central Processing Unit})
  \item[CRS] Sistema de Referência de Coordenadas (\textit{Coordinate Reference System})
  \item[CSS] \textit{Chirp Spread Spectrum} (Espalhamento espectral por chirp)
  \item[CSV] Valores Separados por Vírgula (\textit{Comma-Separated Values})
  \item[DOY] Dia do Ano (\textit{Day of Year})
  \item[DVC] \textit{Data Version Control} (controle de versão de dados)
  \item[EM38-MK2] Sensor de indução eletromagnética Geonics EM38-MK2
  \item[ESP] Família de microcontroladores da Espressif
  \item[ESP32] Microcontrolador/SoC da Espressif (família ESP)
  \item[ET] Evapotranspiração
  \item[F1] Medida-F1 (\textit{F1-score})
  \item[GDAL] \textit{Geospatial Data Abstraction Library}
  \item[GeoTIFF] \textit{Geographic Tagged Image File Format}
  \item[GPIO] Entradas/Saídas de Propósito Geral (\textit{General Purpose Input/Output})
  \item[GPS] Sistema de Posicionamento Global (\textit{Global Positioning System})
  \item[Git] Sistema de controle de versão distribuído (\textit{Git})
  \item[GitHub] Plataforma de hospedagem e colaboração de repositórios Git
  \item[HLS] \textit{Harmonized Landsat and Sentinel-2}
  \item[HTTP] \textit{Hypertext Transfer Protocol}
  \item[I2C] \textit{Inter-Integrated Circuit}
  \item[IDE] \textit{Integrated Development Environment} (Ambiente de Desenvolvimento Integrado)
  \item[INMET] Instituto Nacional de Meteorologia
  \item[IoT] Internet das Coisas (\textit{Internet of Things})
  \item[IoU] Interseção sobre União (\textit{Intersection over Union})
  \item[LPWAN] \textit{Low-Power Wide-Area Network}
  \item[LoRa] \textit{Long Range}
  \item[LoRaWAN] \textit{LoRa Wide Area Network}
  \item[LST] Temperatura da Superfície Terrestre (\textit{Land Surface Temperature})
  \item[LX6] Núcleo Xtensa LX6 (microarquitetura usada no ESP32)
  \item[MAE] Erro Absoluto Médio (\textit{Mean Absolute Error})
  \item[MCU] Unidade de Microcontrolador (\textit{Microcontroller Unit})
  \item[ML] Aprendizado de Máquina (\textit{Machine Learning})
  \item[MNDWI] \textit{Modified Normalized Difference Water Index}
  \item[MQTT] \textit{Message Queuing Telemetry Transport}
  \item[MSI] \textit{MultiSpectral Instrument} (Sentinel-2)
  \item[MySQL] Sistema Gerenciador de Banco de Dados relacional MySQL
  \item[NDMI] \textit{Normalized Difference Moisture Index}
  \item[NDVI] \textit{Normalized Difference Vegetation Index}
  \item[NDWI] \textit{Normalized Difference Water Index}
  \item[NIR] Infravermelho Próximo (\textit{Near-Infrared})
  \item[ODS] Objetivos de Desenvolvimento Sustentável
  \item[OLI/TIRS] \textit{Operational Land Imager}/\textit{Thermal Infrared Sensor} (Landsat 8/9)
  \item[ONU] Organização das Nações Unidas
  \item[PALSAR] \textit{Phased Array type L-band Synthetic Aperture Radar}
  \item[PCA] Análise de Componentes Principais (\textit{Principal Component Analysis})
  \item[QA/QC] Garantia/Controle de Qualidade (\textit{Quality Assurance/Quality Control})
  \item[QGIS] \textit{Quantum GIS} (software SIG de código aberto)
  \item[RF] Rádio Frequência
  \item[RMSE] Raiz do Erro Quadrático Médio (\textit{Root Mean Square Error})
  \item[RTC] Relógio de Tempo Real (\textit{Real-Time Clock})
  \item[S1] Sentinel-1
  \item[S2] Sentinel-2
  \item[SAR] Radar de Abertura Sintética (\textit{Synthetic Aperture Radar})
  \item[SDA/SCL] Linhas de dados/clock do barramento I2C (\textit{Serial Data}/\textit{Serial Clock})
  \item[SIMEPAR] Sistema de Tecnologia e Monitoramento Ambiental do Paraná
  \item[SMA] Sistema Multiagente
  \item[SNAP] \textit{Sentinel Application Platform}
  \item[SoC] Sistema em Chip (\textit{System-on-Chip})
  \item[SPI] \textit{Serial Peripheral Interface}
  \item[SR] Sensoriamento Remoto
  \item[SRAM] Memória RAM Estática (\textit{Static Random-Access Memory})
  \item[SWIR] Infravermelho de Ondas Curtas (\textit{Shortwave Infrared})
  \item[TCC] Trabalho de Conclusão de Curso
  \item[TTN] \textit{The Things Network}
  \item[UART] \textit{Universal Asynchronous Receiver-Transmitter}
  \item[ULA] Unidade Lógica e Aritmética
  \item[UNIFESP] Universidade Federal de São Paulo
  \item[UR] Umidade Relativa
  \item[UTM] \textit{Universal Transverse Mercator}
  \item[Wi-Fi] Rede local sem fio (\textit{Wireless Fidelity})
  \item[ZigBee] Protocolo de comunicação sem fio de baixo consumo (\textit{ZigBee})
\end{siglas}
% ---

% ---
% inserir lista de símbolos
% % ---
% \begin{simbolos}
%   \item[$ m $] Massa do veículo
%   \item[$ x $] Posição do veículo
%   \item[$ \theta $] Ângulo de ataque do veículo
%   \item[$ \Omega $] Ohm
% \end{simbolos}
% % ---

% ---
% inserir o sumario
% ---
\pdfbookmark[0]{\contentsname}{toc}
\tableofcontents*
\cleardoublepage
% ---

% ----------------------------------------------------------
% ELEMENTOS TEXTUAIS
% ----------------------------------------------------------
\textual

% --------------------------------------------------
% Capítulo da Introdução
% --------------------------------------------------
\chapter{Introdução}
\label{cap_Intro}

Nas últimas décadas, o aumento da demanda global por alimentos e a intensificação da
agricultura têm elevado a pressão sobre os recursos naturais, especialmente a água. Esse recurso,
essencial para a manutenção da vida e para a produção agrícola, vem sofrendo escassez em diversas
regiões, o que tem intensificado a busca por soluções que promovam o uso racional e sustentável da
água \cite{fao2022sofa}. Essa preocupação está diretamente alinhada aos Objetivos de Desenvolvimento
Sustentável (ODS), estabelecidos pela Agenda 2030 da Organização das Nações Unidas (ONU), em
especial os ODS de números 2, 6 e 12, que visam assegurar a segurança alimentar, melhorar o
manejo hídrico e promover práticas agrícolas sustentáveis. Nesse contexto, a gestão eficaz e
sustentável da água na agricultura não é apenas uma necessidade técnica, mas uma demanda
decisiva para garantir a estabilidade ambiental e social \cite{rocha2024exploracao}.

Ao mesmo tempo, o avanço das tecnologias digitais aplicadas à agricultura, como o
sensoriamento remoto e a IoT, tem possibilitado a coleta massiva de dados
ambientais e produtivos, abrindo novas oportunidades para o aprimoramento do manejo hídrico e o
aumento da eficiência no uso da água. Entretanto, observa-se que a adoção dessas tecnologias ainda
ocorre de forma fragmentada, impedindo uma visão integrada e em tempo quase real da dinâmica
hídrica nas lavouras. Segundo \citeonline{wang2024integration}, apesar do progresso obtido pela combinação de
sensores de solo, dados de satélite e técnicas de aprendizado de máquina, persistem desafios
relacionados à qualidade, à padronização e à interoperabilidade das informações — fatores que
comprometem a construção de modelos generalizáveis para a agricultura de precisão. Essa falta de
integração entre diferentes fontes de dados limita a geração de produtos aplicáveis, como mapas de
irrigação e relatórios de recomendação, que são essenciais para orientar práticas agrícolas mais
eficientes e ambientalmente responsáveis.

Diante desse cenário, o presente estudo propõe o desenvolvimento de uma metodologia de
fusão de dados que combine informações provenientes de sensores IoT e de imagens de satélite,
utilizando técnicas de aprendizado de máquina para gerar recomendações precisas e dinâmicas de
irrigação. Essa abordagem busca superar as lacunas existentes na integração e automação de dados
agrícolas, promovendo uma visão mais completa e multiescalar do comportamento hídrico das áreas
cultivadas. Ao conectar o desafio global da sustentabilidade hídrica às oportunidades
proporcionadas pela inovação tecnológica, esta pesquisa se apresenta como uma contribuição
concreta à agricultura de precisão sustentável, fortalecendo o compromisso com o uso racional dos
recursos naturais e com os objetivos propostos pela Agenda 2030.

% ---
% Seção Trabalhos Correlatos
% ---
\section{Trabalhos Correlatos}
\label{sec:trabalhos_correlatos}

Após a contextualização apresentada, que evidencia a pressão crescente sobre os recursos
hídricos e o alinhamento do tema aos Objetivos de Desenvolvimento Sustentável (ODS 2, 6 e
12), e de toda fundamentação teórica, que definiu os conceitos de IoT e
sensoriamento remoto aplicados ao manejo hídrico, esta seção discute trabalhos que
efetivamente aplicam essas tecnologias em contextos agrícolas.

Os estudos selecionados foram organizados em três eixos: (i) IoT na agricultura, desde revisões
gerais até aplicações em campo; (ii) sensoriamento remoto e gestão hídrica; e (iii) fusão de
sensores próximos e remotos. Em conjunto, esses trabalhos ajudam a explicitar a lacuna que
esta monografia busca abordar: a integração sistemática entre dados de redes de sensores IoT
e produtos de sensoriamento remoto para apoiar decisões de irrigação em agricultura de
precisão.

\subsection{IoT na agricultura: de revisões gerais a aplicações em campo}

Em consonância com a discussão sobre o papel da IoT na agricultura 4.0,
\citeonline{kumar2024smart} apresentam uma revisão extensa sobre o uso de tecnologias de IoT
na agricultura inteligente e sustentável. Os autores descrevem arquiteturas em camadas
(sensoriamento, comunicação e aplicação), tipos de sensores (solo, clima, planta), protocolos de
comunicação (Wi-Fi, LoRa, ZigBee, entre outros) e plataformas em nuvem para armazenamento
e análise de dados. O survey também destaca aplicações como irrigação inteligente,
monitoramento de condições ambientais, rastreabilidade e automação de operações agrícolas,
associando-as a ganhos em eficiência de uso de insumos, produtividade e sustentabilidade. Ao
mesmo tempo, aponta desafios recorrentes, como limitações de conectividade em áreas rurais,
interoperabilidade entre dispositivos heterogêneos, segurança, privacidade e modelos de
negócio para adoção em larga escala.

Focando especificamente em sistemas pecuários, mas com pertinência técnica ao contexto agrícola
em geral, \citeonline{farooq2022survey} realizam um survey sobre o papel da IoT na
implementação de ambientes de pecuária inteligente. O trabalho discute de forma detalhada a
infraestrutura de redes IoT, arquiteturas em camadas, topologias e protocolos de
comunicação, bem como desafios de escalabilidade, consumo energético, robustez da
comunicação sem fio e padronização. São apresentados casos de uso envolvendo sensores
vestíveis, coleiras inteligentes e nós distribuídos para monitoramento de saúde, bem-estar e
produtividade dos animais. Embora o foco seja a produção animal, os pontos críticos
levantados -- sobretudo no que se refere à confiabilidade da rede e à gestão de grandes
volumes de dados -- são análogos aos enfrentados por redes de sensores de solo e clima,
reforçando aspectos sobre os limites e potencialidades da IoT em ambientes rurais.

No contexto da agricultura familiar brasileira, \citeonline{gomes2023iot} propõem uma solução
de baixo custo baseada em IoT para monitoramento agroclimático e suporte à irrigação
sustentável, aproximando os conceitos discutidos na revisão de \citeonline{kumar2024smart} de
uma realidade de pequena escala, fundamental para o presente estudo. O sistema integra
sensores de umidade e temperatura do ar e do solo, sensores de vazão e atuadores de irrigação
a um software supervisório de código aberto (ScadaBR), com armazenamento dos dados em
banco MySQL. Implantada em horta experimental no Sul/Sudoeste de Minas Gerais, a solução
apresenta boa concordância entre os dados coletados e informações meteorológicas oficiais do
INMET, evidenciando potencial para apoiar decisões de irrigação e reduzir desperdícios de
água em propriedades com restrições de capital. Este estudo reforça, de
forma prática, a viabilidade de arquiteturas acessíveis e abertas para o contexto de agricultura
familiar e para estudo em pequeno porte.

\citeonline{trinta2021gerenciamento} avançam na direção da automatização da irrigação, ao
apresentar um sistema autônomo de gerenciamento baseado em uma arquitetura com Sistema
Multiagente (SMA) e IoT. Na solução proposta, um Raspberry Pi executa os agentes
responsáveis por tomar decisões a partir de dados coletados por microcontroladores, que leem
sensores de solo e clima e acionam atuadores de irrigação. A comunicação ocorre via
\textit{middleware} ContextNet, e uma aplicação web permite ao usuário acompanhar o estado do
cultivo e intervir quando necessário. Os resultados do estudo de caso indicam que a abordagem
com SMA aumenta a autonomia do sistema, possibilitando decisões de irrigação mesmo na
ausência de intervenção humana direta e reduzindo a dependência de monitoramento manual. Ao trazer a inteligência para a borda da rede, o trabalho se
aproxima da proposta desta monografia de utilizar dados de sensores de campo como insumo
para decisões automatizadas de manejo hídrico, porém ainda sem integrar explicitamente
informações orbitais.

\subsection{Sensoriamento remoto e gestão hídrica na agricultura}

Do ponto de vista da gestão hídrica em escala de sistema produtivo,
\citeonline{rocha2024exploracao} realizam uma revisão sobre técnicas de irrigação e
tecnologias voltadas à minimização do desperdício de água na agricultura, retomando a
problemática de escassez de recursos. O artigo examina alguns métodos de irrigação (como
gotejamento, aspersão e sistemas localizados), além de tecnologias de monitoramento que
incluem desde sensores de solo, estações meteorológicas até ferramentas de apoio às decisões.
Os autores discutem ganhos em economia de água, energia e fertilizantes, mas ressaltam
barreiras relacionadas a custos de implantação, necessidade de capacitação técnica e
desigualdade de acesso entre os produtores. Ao defenderem uma abordagem integrada que
considere simultaneamente dimensões técnicas, econômicas, sociais e ambientais, reforçam a
importância de arranjos tecnológicos como o proposto nesta monografia, que articula IoT e
sensoriamento remoto.

\citeonline{muturi2025review} acrescentam a perspectiva do sensoriamento remoto, ao
revisarem o uso de técnicas orbitais para estimar o uso de água na irrigação. O estudo mapeia
abordagens baseadas em evapotranspiração, umidade do solo, balanço hídrico e fusão de dados
de múltiplos sensores, contemplando diferentes escalas espaciais. São discutidos o papel de
sensores ópticos, térmicos e de micro-ondas, bem como a importância de séries temporais e de
dados de campo para calibração e validação. Entre as lacunas apontadas estão a subexploração
de sensores de micro-ondas, a escassez de séries longas e de estudos comparativos, além da
necessidade de quantificar incertezas quando essas estimativas subsidiam políticas de alocação
de água. Essas observações reforçam a relevância de integrar
medições locais, como as fornecidas por redes IoT, às estimativas derivadas de imagens de
satélite.

\citeonline{wang2024integration} ampliam esse panorama ao discutir a integração entre
sensoriamento remoto e algoritmos de aprendizado de máquina na agricultura de precisão,
retomando a ênfase no uso de técnicas de inteligência computacional. Os autores sistematizam
aplicações que combinam imagens de satélite, sensores hiperespectrais, plataformas aéreas não
tripuladas e sensores de proximidade com modelos como máquinas de vetor de
suporte, florestas aleatórias e redes neurais profundas. São apresentados exemplos em
estimativa de produtividade, detecção de estresse hídrico, mapeamento de atributos de solo e
delimitação de zonas de manejo. Ao mesmo tempo, o estudo enfatiza desafios de qualidade e
padronização dos dados, interpretabilidade dos modelos e transferência entre regiões com
diferentes condições de solo e de clima. Esses pontos se alinham
com as limitações de generalização e importância do contexto local, reforçando a necessidade
de abordagens de fusão que aproveitem dados orbitais e de campo de forma complementar.

\subsection{Fusão de sensores próximos e remotos no manejo da irrigação}

\citeonline{rodrigues2024digitaltwins} aproximam-se de forma direta da proposta desta
monografia ao explorar a fusão de dados de sensores proximais e remotos para apoio ao
zoneamento de manejo de irrigação, articulada ao conceito de ``gêmeos digitais''. Em um pivô
central de 72 ha no estado de São Paulo, os autores coletam dados de condutividade elétrica
aparente do solo com sensor proximal EM38-MK2 em alta densidade e, em seguida, simulam
um cenário de amostragem esparsa com menos linhas de coleta, representando uma situação de
menor custo operacional. Esses dados são combinados com covariáveis derivadas de modelos
digitais de elevação e de imagens dos satélites \textit{Advanced Land Observing Satellite} (ALOS) PALSAR, \textit{Advanced Spaceborne Thermal Emission and Reflection Radiometer} (ASTER), Sentinel-2 e Landsat
8. Comparando diferentes métodos geoestatísticos e de regressão espacial, o estudo evidencia
que a krigagem com deriva externa, alimentada pelas covariáveis de sensoriamento remoto,
permite aproximar a acurácia do mapa de referência obtido com amostragem densa, tornando
possível definir zonas de manejo de irrigação com menor esforço de campo.

Ainda que não utilizem explicitamente redes IoT, os trabalhos de \citeonline{muturi2025review},
\citeonline{wang2024integration} e \citeonline{rodrigues2024digitaltwins} convergem com a
Fundamentação Teórica ao demonstrar que o sensoriamento remoto, combinado a medições
locais, é capaz de gerar produtos espaciais -- como mapas de uso de água, zonas de manejo e
indicadores de estresse hídrico -- altamente relevantes para decisões de irrigação. O ponto de
tensão que emerge desses estudos, e que se conecta diretamente à lacuna identificada na
literatura, é a ausência de arquiteturas bem descritas que integrem de ponta a ponta séries
temporais de sensores e informações orbitais em fluxos operacionais de apoio à decisão.

\subsection{Síntese crítica e relação com o trabalho proposto}

Os trabalhos analisados confirmam o panorama delineado: soluções baseadas em IoT evoluíram
significativamente rumo ao monitoramento em tempo quase real e à automação da irrigação em
nível de propriedade, seja por meio de arquiteturas de baixo custo voltadas à agricultura
familiar \cite{gomes2023iot}, seja via sistemas com inteligência distribuída, como os baseados
em sistemas multiagentes \cite{trinta2021gerenciamento}. Em paralelo, a literatura em
sensoriamento remoto e aprendizado de máquina consolidou métodos para estimar
evapotranspiração, umidade do solo e uso de água na irrigação, além de delinear zonas de
manejo com base em dados orbitais e proximais
\cite{rocha2024exploracao,muturi2025review,wang2024integration,rodrigues2024digitaltwins}.

No entanto, a maior parte desses estudos permanece concentrada em um desses eixos de forma
isolada: ora enfatizam a infraestrutura de IoT e a lógica de decisão local, ora exploram o
potencial do sensoriamento remoto e de modelos preditivos, sem detalhar metodologias de
fusão que conciliem a alta frequência temporal dos sensores de campo com a cobertura espacial
e multiespectral das imagens de satélite. Assim, embora haja indicações claras do benefício de
combinar essas fontes de dados -- tanto nas revisões quanto nos estudos aplicados -- ainda são
relativamente raras as propostas que implementam, validam e documentam arquiteturas
completas de integração entre IoT e sensoriamento remoto para apoio operacional à gestão
hídrica em escala de talhão.

Diante desse panorama, reafirma-se o espaço de contribuição desta monografia: desenvolver e
avaliar um método de fusão de dados entre redes de sensores IoT e produtos de sensoriamento
remoto, apoiado em técnicas de aprendizado de máquina, voltado à otimização da gestão
hídrica na agricultura de precisão. Ao articular os elementos discutidos na Introdução, na
Fundamentação Teórica e nos Trabalhos Correlatos, o estudo busca oferecer uma abordagem
integrada que contribua para o uso mais eficiente e sustentável da água, em apoio às diretrizes
dos ODS e aos desafios práticos enfrentados em sistemas agrícolas reais.

% ---
% Seção Definição do Problema
% ---
\section{Definição do Problema}

A agricultura é um dos setores que mais demandam água e, ao mesmo tempo, um dos mais
sensíveis à sua disponibilidade e à variabilidade espacial e temporal no campo. Na prática, ainda é
comum que decisões de irrigação sejam tomadas com base em cronogramas fixos, observação
empírica ou medições pontuais, o que pode levar tanto ao desperdício de água quanto ao estresse
hídrico das culturas. Em um cenário de crescente pressão sobre os recursos naturais, essa lacuna
dificulta a adoção de estratégias de manejo mais racionais e sustentáveis, alinhadas às demandas de
eficiência e aos ODS relacionados a produção agrícola e gestão hídrica.

Embora tecnologias como sensores IoT e imagens de satélite permitam observar o sistema agrícola
em diferentes escalas, sua adoção frequentemente ocorre de forma fragmentada, com dados
heterogêneos, diferentes resoluções e dificuldades de interoperabilidade e sincronização temporal.
Como consequência, torna-se desafiador transformar medições em campo e produtos de
sensoriamento remoto em informações integradas e acionáveis, como mapas e recomendações
dinâmicas para o manejo hídrico. Assim, o problema central abordado neste trabalho é a ausência
de uma metodologia e de uma arquitetura integradas que permitam fundir, padronizar e explorar
essas fontes de dados de maneira consistente, viabilizando produtos aplicáveis para a agricultura de
precisão.


% ---
% Seção Objetivos
% ---
\section{Objetivos}
Esta seção apresenta os objetivos que orientam o desenvolvimento do trabalho. O objetivo geral
sintetiza a principal contribuição esperada com a proposta, enquanto os objetivos específicos
detalham as etapas e entregáveis necessários para alcançá-la, incluindo a definição de requisitos,
a arquitetura do sistema, o tratamento e a fusão das fontes de dados e a avaliação dos produtos
gerados.

\subsection{Objetivo Geral}
O objetivo geral deste trabalho é propor e desenvolver um sistema de agricultura de precisão
focado no manejo hídrico sustentável, baseado na fusão de dados de sensores IoT e imagens
de satélite, visando gerar mapas de recomendação de irrigação mais precisos e
eficientes.

\subsection{Objetivos Específicos}
Para atingir o objetivo geral proposto, estabelecem-se os seguintes objetivos específicos:

\begin{itemize}
    \item Revisar a literatura e o estado da arte sobre as tecnologias de IoT, sensoriamento remoto
    e técnicas de fusão de dados aplicadas à agricultura de precisão;

    \item Definir a arquitetura do sistema, especificando os componentes de hardware para a
    coleta de dados e a plataforma de software para ingestão e armazenamento dos dados;

    \item Definir e estabelecer a metodologia para a aquisição e o processamento de imagens de
    satélite para a área de estudo;

    \item Desenvolver um modelo de \textit{machine learning} para realizar a fusão dos dados,
    correlacionando as medições pontuais e de alta frequência dos sensores IoT com os dados
    espaciais de baixa frequência das imagens de satélite;

    \item Implementar um protótipo do sistema que utilize o modelo de fusão para gerar mapas,
    relatórios ou \textit{dashboards} de recomendação e alerta sobre a irrigação, indicando a
    variabilidade da necessidade de irrigamento;

    \item Analisar a eficácia potencial do sistema proposto na redução do desperdício de água e
    no aumento da eficiência produtiva, validando a metodologia adotada.
\end{itemize}

% ---
% Seção Estrutura do Texto
% ---
\section{Estrutura do Texto}

Este trabalho está organizado em cinco capítulos, de modo a conduzir o leitor desde a contextualização do 
problema até a proposição da solução e a análise dos resultados. No Capítulo~\ref{cap_Intro}, apresentam-se a motivação e o 
contexto relacionados ao manejo sustentável de recursos hídricos na agricultura, bem como a definição do problema, os 
objetivos do estudo e a síntese de trabalhos correlatos que fundamentam a proposta.

Os capítulos subsequentes são estruturados da seguinte forma:
\begin{itemize}
    \item O Capítulo~\ref{cap_fundTeo} reúne os principais conceitos necessários para o desenvolvimento do trabalho, abrangendo 
    Internet das Coisas e protocolos de comunicação aplicados ao monitoramento em campo, aspectos de \textit{hardware} e 
    sistemas embarcados, fundamentos de sensoriamento remoto e índices espectrais, além de noções de aprendizado de máquina e 
    estratégias de integração/fusão de dados.
    \item O Capítulo~\ref{cap_desenv} descreve os materiais e métodos empregados, contemplando as fontes de dados (uso de \textit{datasets} e 
    coleta em campo com rede baseada em LoRaWAN), a arquitetura proposta do sistema e o fluxo de aquisição, transmissão, 
    armazenamento e pré-processamento, incluindo procedimentos de sincronização temporal e definição de atributos para análise.
    \item O Capítulo~\ref{cap_results} apresenta os resultados obtidos e a respectiva discussão, analisando o comportamento do sistema 
    e do pipeline de dados, bem como suas limitações e implicações para o manejo hídrico em agricultura de precisão.
    \item Por fim, o Capítulo~\ref{cap_conclusao} consolida as conclusões do estudo, relacionando-as aos objetivos propostos, e indica 
    perspectivas de trabalhos futuros, como a ampliação da validação em campo, a incorporação de novos sensores/produtos orbitais e o 
    aprimoramento das estratégias de fusão de dados.
\end{itemize}


  
% ---
% Capitulo de revisão de literatura
% ---
\chapter{Fundamentação Teórica}
\label{cap_fundTeo}

Este capítulo estabelece os pilares técnicos do projeto, detalhando a infraestrutura de comunicação e \textit{hardware} (IoT), os princípios físicos do sensoriamento remoto e os métodos computacionais de aprendizado de máquina utilizados para a fusão de dados.

% ---------------------------------------------------------
% SEÇÃO 2.1 - INTERNET DAS COISAS (IoT)
% ---------------------------------------------------------
\section{Internet das Coisas e Protocolos de Comunicação}
\label{sec:iot}

A IoT é uma subárea da computação que se dedica à interconexão de dispositivos físicos à Internet, permitindo a coleta e troca de dados em tempo real. Seu funcionamento é pautado na integração de sensores, protocolos de comunicação e plataformas de processamento, e os produtos gerados dessa interação são fundamentais para a diversas áreas modernas, como cidades inteligentes, saúde conectada e agricultura de precisão \cite{maschietto2021arquitetura}.

Quanto à agricultura de precisão, IoT transcende a simples conectividade de dispositivos, exigindo uma compreensão detalhada de arquiteturas distribuídas e restrições de energia. Segundo \citeonline{kumar2024smart}, uma solução de IoT robusta deve ser estruturada em camadas funcionais bem definidas: percepção (sensores), rede (transporte) e aplicação (processamento). Um esquemático de uma rede IoT em alto nível é apresentado na Figura \ref{fig:iot_architecture}.

\begin{figure}[ht]
    \centering
    \caption{Arquitetura de alto nível para rede IoT.}
    \includegraphics[width=0.7\textwidth]{figures/High-level-IoT-architecture.jpg}
    \legend{Fonte: \cite{hakiri2015iot}.}
    \label{fig:iot_architecture}
\end{figure}

A Figura \ref{fig:iot_architecture} ilustra a estrutura típica de uma rede IoT, composta por três camadas principais: camada de domínio do dispositivo (sensores e atuadores), camada de rede (transporte de dados) e camada de aplicação (processamento e análise). Cada camada desempenha um papel crucial na coleta, transmissão e utilização dos dados, e a escolha dos componentes e protocolos adequados é fundamental para o sucesso do sistema \cite{hakiri2015iot}.

\subsection{Redes LPWAN e Protocolo LoRaWAN}
Para o transporte de dados em cenários agrícolas, onde a infraestrutura celular é frequentemente precária, utilizam-se redes LPWAN (\textit{Low-Power Wide-Area Networks}). Diferente de redes locais (Wi-Fi, ZigBee), as LPWAN priorizam o alcance estendido (quilômetros) e a eficiência energética em detrimento da alta taxa de transferência.

Neste trabalho, adota-se a tecnologia \textbf{LoRa} (\textit{Long Range}) na camada física. A robustez do LoRa advém de sua modulação baseada em \textit{Chirp Spread Spectrum} (CSS), que permite a decodificação de sinais mesmo abaixo do nível de ruído \cite{farooq2021iotagriculture}. Sobre ela, opera o protocolo LoRaWAN (camada MAC), que define uma topologia em estrela onde os nós sensores comunicam-se diretamente com \textit{gateways} centrais, otimizando o ciclo de vida das baterias.

% ---------------------------------------------------------
% SEÇÃO 2.2 - HARDWARE E SISTEMAS EMBARCADOS
% ---------------------------------------------------------
\section{Hardware e Sistemas Embarcados}
\label{sec:hardware_embedded}

A implementação de nós sensores em campo exige a compreensão da arquitetura de sistemas embarcados, que difere da computação de propósito geral por restrições severas de energia, memória e tempo de resposta.

\subsection{Microcontroladores (MCUs) e o SoC ESP32}
Diferente dos microprocessadores convencionais, que dependem de barramentos externos para acessar memória e periféricos, os microcontroladores (\textit{Microcontroller Units} - MCUs) são dispositivos altamente integrados (\textit{System on Chip} - SoC). Um MCU típico encapsula em uma única pastilha de silício a Unidade Central de Processamento (CPU), memória não-volátil (Flash) para armazenamento do \textit{firmware}, memória volátil (SRAM) para dados e interfaces de Entrada/Saída (GPIO) \cite{maschietto2021arquitetura}.

No contexto deste trabalho, a escolha de arquiteturas de 32 bits, especificamente o ESP32, justifica-se pela necessidade de processamento concorrente: o dispositivo deve gerenciar a leitura de sensores via interrupções de hardware enquanto mantém ativa a pilha de protocolos de comunicação. Ademais, a capacidade de operar o protocolo LoRa fez com que a plataforma ESP32 se destacasse frente a MCUs tradicionais de 8 bits (como Arduino).

A arquitetura interna do ESP32 é apresentada na Figura \ref{fig:esp32_block}. O diagrama evidencia a integração de dois núcleos de processamento (Dual-Core Xtensa LX6 32-bit), módulos de rádio (Wi-Fi/Bluetooth) e periféricos críticos como o RTC (\textit{Real-Time Clock}), que permite o gerenciamento avançado de energia (\textit{Deep Sleep}). 

\begin{figure}[ht]
    \centering
    \caption{Diagrama de blocos funcional do SoC ESP32.}
    \includegraphics[width=0.7\textwidth]{figures/block_diagram_esp32.png}
    \legend{Fonte: Espressif Systems (Datasheet oficial).}
    \label{fig:esp32_block}
\end{figure}

A interface com o mundo físico ocorre através do encapsulamento e distribuição de pinos (Figura \ref{fig:esp32_pinout}). A maioria dos pinos possui função multiplexada, podendo ser configurados via \textit{software} como entradas digitais, canais de conversão ADC ou linhas de comunicação serial. Essa flexibilidade é um outro fator que torna o ESP32 adequado para aplicações IoT, onde a diversidade de sensores e atuadores exige múltiplas interfaces.

\begin{figure}[ht]
    \centering
    \caption{Layout de pinos do encapsulamento do ESP32.}
    \includegraphics[width=0.7\textwidth]{figures/pin_layout_esp32.png}
    \legend{Fonte: Espressif Systems (Datasheet oficial).}
    \label{fig:esp32_pinout}
\end{figure}

\subsection{Princípios de Transdução e Sensores}
Sensores são dispositivos transdutores que convertem grandezas físicas em sinais elétricos mensuráveis. No contexto agrícola, sensores de umidade do solo são cruciais para monitorar a disponibilidade hídrica às plantas. Esses sensores podem ser classificados em duas categorias principais:

\begin{itemize}
    \item \textbf{Sensores Analógicos:} Retornam uma tensão variável ($0$ a $3.3V$) que deve ser discretizada pelo Conversor Analógico-Digital (ADC) do microcontrolador, presente no diagrama de blocos do ESP32 (Figura \ref{fig:esp32_block}).
    \item \textbf{Sensores Digitais:} Possuem circuitos integrados que realizam a conversão internamente e transmitem o dado via barramento serial.
\end{itemize}

\subsection{Interfaces de Comunicação Periférica}
A comunicação interna no nó sensor (entre o MCU e seus sensores ou transceptores de rádio) ocorre através de protocolos seriais:

\begin{itemize}
    \item \textbf{UART (\textit{Universal Asynchronous Receiver-Transmitter}):} Protocolo assíncrono ponto-a-ponto, fundamental para depuração e comunicação com módulos GPS.
    \item \textbf{I2C (\textit{Inter-Integrated Circuit}):} Barramento síncrono de dois fios (SDA/SCL), utilizado para conectar múltiplos sensores digitais.
    \item \textbf{SPI (\textit{Serial Peripheral Interface}):} Protocolo síncrono de alta velocidade (\textit{full-duplex}), essencial para a comunicação com o transceptor LoRa, que exige alta vazão de dados e determinismo temporal.
\end{itemize}

% ---------------------------------------------------------
% SEÇÃO 2.3 - SENSORIAMENTO REMOTO
% ---------------------------------------------------------
\section{Sensoriamento Remoto (SR)}
\label{sec:sr_basico}

O SR baseia-se na interação da radiação eletromagnética com a superfície terrestre. Para compreender a extração de informações das imagens, é necessário analisar o processo físico de aquisição. Com isso, SR se trata de uma área da geociência, engenharia e ciências ambientais, que estuda a obtenção de informações sobre objetos ou áreas à distância, geralmente por meio de satélites ou aeronaves equipadas com sensores especializados.

\subsection{Processo Físico de Formação da Imagem}
A geração de uma imagem de satélite passivo óptico (como Sentinel-2) segue um fluxo físico complexo:

\begin{enumerate}
    \item \textbf{Fonte e Interação:} O Sol emite radiação que atravessa a atmosfera e incide sobre o alvo. Parte dessa energia é absorvida, transmitida ou refletida em direção ao sensor.
    \item \textbf{Radiância no Topo da Atmosfera (TOA):} O sensor mede a radiância total ($L$), que inclui a luz refletida pelo alvo e a dispersão atmosférica.
    \item \textbf{Conversão Radiométrica e Atmosférica:} Para análise temporal, aplica-se a \textbf{Correção Atmosférica}, convertendo os dados brutos em \textbf{Refletância de Superfície (BOA - \textit{Bottom of Atmosphere})}, removendo efeitos de aerossóis e vapor d'água \cite{claverie2018hls}.
\end{enumerate}

% Adicionando uma figura que mostra o funcionamento da formação da imagem por SR por radiância do sol
\begin{figure}[ht]
    \centering
    \caption{Processo físico de formação da imagem de sensoriamento remoto pela fonte solar.}
    \includegraphics[width=0.7\textwidth]{figures/passive_SR_instruments.png}
    \legend{Fonte: \cite{nasa2023spectral}.}
    \label{fig:sr_image_formation}
\end{figure}

A Figura \ref{fig:sr_image_formation} ilustra o processo de formação da imagem em sensores passivos, destacando a interação da radiação solar com a atmosfera e a superfície terrestre. Os sensores são calibrados para medir a radiância refletida em diferentes bandas espectrais, permitindo a análise detalhada das características do alvo. No entando, esse processo é suscetível a interferências atmosféricas, o que reforça a importância da correção atmosférica para garantir a precisão dos dados obtidos \cite{nasa2023spectral}.

\subsection{Índices Espectrais de Vegetação e Água}
Os índices espectrais são operações algébricas entre bandas, projetadas para realçar características de interesse \cite{tran2022reviewspectral}.

\subsubsection{NDVI (\textit{Normalized Difference Vegetation Index})}
Indicador consolidado para vigor vegetativo, baseado no contraste entre a absorção de clorofila no Vermelho ($R$) e a reflectância da estrutura celular no Infravermelho Próximo ($NIR$). Seu valor é calculado pela equação:

\begin{equation}
    NDVI = \frac{\rho_{NIR} - \rho_{R}}{\rho_{NIR} + \rho_{R}}
\end{equation}

A variação do NDVI ao longo do tempo permite monitorar o crescimento das culturas e identificar áreas de estresse hídrico. Portanto, é um índice fundamental para a agricultura de precisão e para o manejo hídrico eficiente.

\subsubsection{NDWI (\textit{Normalized Difference Water Index})}
Visa delinear corpos hídricos ou estresse na vegetação, utilizando os valores das bandas verde e o infravermelho, conforme a fórmula:

\begin{equation}
    NDWI = \frac{\rho_{Green} - \rho_{NIR}}{\rho_{Green} + \rho_{NIR}}
\end{equation}

Valores positivos indicam presença de água ou alta umidade, enquanto valores negativos sugerem solo seco ou vegetação estressada. Variações como o MNDWI substituem o NIR pelo SWIR para maior precisão em certas condições. No entanto, esse índice é sensível a interferências atmosféricas, o que reforça a importância da correção atmosférica para garantir a precisão dos dados obtidos.

% ---------------------------------------------------------
% SEÇÃO 2.4 - MACHINE LEARNING
% ---------------------------------------------------------
\section{Aprendizado de Máquina (\textit{Machine Learning})}
\label{sec:ml}

O Aprendizado de Máquina (ML) é o subcampo da Inteligência Artificial que tem como objetivo o desenvolvimento de algoritmos capazes de aprender padrões a partir de dados, sem a necessidade de programação explícita para cada tarefa específica. Esses algoritmos são treinados em conjuntos de dados para identificar relações complexas e fazer previsões ou classificações com base em novas entradas.

\subsection{Tipos de Aprendizado}
\begin{itemize}
    \item \textbf{Aprendizado Supervisionado:} O algoritmo é treinado com dados rotulados (entrada e saída conhecida) para aprender uma função de mapeamento $y = f(x)$.
    \item \textbf{Aprendizado Semi-Supervisionado:} Combina dados rotulados e não rotulados para melhorar o desempenho do modelo quando a rotulagem é cara ou limitada.
    \item \textbf{Aprendizado Não-Supervisionado:} O algoritmo busca estruturas ocultas em dados não rotulados, como o agrupamento (\textit{clustering}) de áreas similares.
    \item \textbf{Aprendizado por Reforço:} O agente aprende a tomar decisões sequenciais através de recompensas ou penalidades, útil para otimização de estratégias ao longo do tempo.
\end{itemize}

\subsection{Tarefas e Fusão de Dados}
No tangente à fusão de dados de IoT e sensoriamento remoto, as tarefas de ML mais relevantes incluem:

\begin{itemize}
    \item \textbf{Regressão:} Modelagem de variáveis contínuas, a partir de características extraídas dos dados. No contexto deste trabalho, a regressão pode ser utilizada para estimar a umidade do solo com base em índices espectrais e dados de sensores IoT.
    \item \textbf{Classificação:} Identificação de estados discretos, para conseguir classificar um dado. Na tentativa de inserir classificação ao escopo deste trabalho, descata-se na aplicação dos níveis de estresse hídrico (baixo, médio, alto) com base nas múltiplas fontes de dados.
    \item \textbf{Redução de Dimensionalidade:} Técnicas como PCA (\textit{Principal Component Analysis}) para extrair as características mais relevantes e reduzir o ruído nos dados.
\end{itemize}

A abordagem de ML permite integrar efetivamente as diferentes resoluções espaço-temporais e características dos dados de IoT e sensoriamento remoto, gerando modelos preditivos robustos para o manejo hídrico.

% ---
% Capitulo de Desenvolvimento
% ---
\chapter{Desenvolvimento}
\label{cap_desenv}

Esta seção descreve os materiais e os métodos a serem empregados no estudo, considerando dois caminhos complementares de obtenção de dados: (A) uso de \textit{datasets} existentes e (B) coleta em campo com rede própria baseada em LoRaWAN. Ao final, apresenta-se o cronograma das atividades planejadas.

% -------------------------
% MATERIAIS
% -------------------------
\section{Materiais}
\label{subsec:materiais}
Pensando nos materiais utilizados para o desenvolvimento, devido à incerteza na extração dos dados utilizados para o estudo, pensou-se em utilizar duas temáticas: dados e \textit{hardware}.

\subsection{Dados}
\label{subsubsec:dados}
A utilização dos dados nesse estudo é fundamental. Ter um bom conjunto de dados para fazer a sua separação para treinar e testar o modelo desenvolvido, terá relação direta no impacto deste projeto. No entanto, devido ao impedimento de ir a campo para coletar as amostras, pensou-se em dois caminhos.

\noindent\textbf{Caminho A — \textit{Dataset} existente}
\begin{itemize}
    \item Séries históricas e/ou \textit{datasets} acadêmicos contendo variáveis de solo/umidade e apoio meteorológico;
    \item Meteorologia (INMET/SIMEPAR ou estação local);
    \item Sensoriamento remoto: Sentinel-2 (MSI), Landsat-8/9 (OLI/TIRS) e Sentinel-1 (SAR);
    \item Dados vetoriais: limites de talhão, rede de drenagem, solos e corpos d’água.
\end{itemize}

\noindent\textbf{Caminho B — Coleta em campo (rede própria)}
\begin{itemize}
    \item Amostra piloto em 1--2 talhões, com medições de umidade do solo e temperatura/umidade do ar;
    \item Registros de validação manual (checklist de campo; amostras pontuais).
\end{itemize}

\subsection{\textit{Hardware} e comunicação (Caminho B)}
\label{subsubsec:hardware}

\begin{itemize}
    \item Microcontrolador \textbf{ESP32} com suporte a \textbf{LoRa};
    \item \textbf{\textit{Gateway} LoRaWAN} (TTN/ChirpStack) ou concentrador disponível;
    \item Sensores: umidade do solo (capacitivo/tensiométrico), temperatur e umidade relativa do ar;
    \item Antenas, caixa com proteção de entrada, cabeamento, fonte/bateria.
\end{itemize}

\subsection{\textit{Software} e serviços}
\label{subsubsec:software}

\begin{itemize}
    \item Ambiente Python (venv/conda); bibliotecas: GDAL/rasterio, SNAP/Sen2Cor, ou ainda Google Earth Engine;
    \item Telemetria (Caminho B): The Things Stack/TTN, MQTT/HTTP, \textit{scripts} de ingestão;
    \item Ciência de dados e visualização: \texttt{numpy}, \texttt{pandas}, \texttt{scikit-learn}, QGIS, \texttt{dash} ou \texttt{streamlit}.
\end{itemize}

\subsection{Organização e gestão de dados}
\label{subsubsec:organizacao}

\begin{itemize}
    \item Estrutura de pastas: \texttt{raw/}, \texttt{interim/}, \texttt{processed/}, \texttt{models/}, \texttt{reports/};
    \item Metadados (CSV/Parquet): \textit{sensor}, \textit{lat}, \textit{lon}, data/hora local, DOY, \texttt{QA flags}, \texttt{CRS};
    \item Controle de versão: Git/GitHub e possivelmente \textit{Data Version Control} (DVC) para grandes volumes de dados.
\end{itemize}


% -------------------------
% MÉTODOS
% -------------------------
\section{Métodos}
\label{subsec:metodos}

\subsection{\textit{Gate} de decisão (Mês 1--2)}
\label{subsubsec:gate}

\begin{enumerate}
    \item Verificar disponibilidade e suficiência de \textit{datasets} existentes (variáveis, período, frequência);
    \item Se suficiente: adotar \textbf{Caminho A} para acelerar a modelagem; caso contrário, ativar \textbf{Caminho B} para coleta em campo com LoRaWAN.
\end{enumerate}

\subsection{Delineamento experimental}
\label{subsubsec:delineamento}

\begin{itemize}
    \item Área de estudo (município/UTM/bioma) e talhões;
    \item Período de análise (safra/estação);
    \item Variáveis mínimas: umidade do solo, T/UR do ar;
    \item Unidades de análise: pixel (10--30~m), ponto de sensor e talhão.
\end{itemize}

\subsection{Aquisição de dados}
\label{subsubsec:aquisicao}

\noindent\textbf{Caminho A — Dataset existente}
\begin{itemize}
    \item Curadoria: fonte, licença, cobertura temporal/espacial, variáveis e qualidade;
    \item Harmonização: fuso/horário local, DOY e unidades;
    \item Integração de meteorologia e vetoriais do talhão.
\end{itemize}

\noindent\textbf{Caminho B — Coleta em campo (ESP32 + LoRaWAN)}
\begin{itemize}
    \item Topologia: nós ESP+LoRa $\rightarrow$ gateway LoRaWAN $\rightarrow$ servidor (TTN/ChirpStack);
    \item \textit{Payload}: \texttt{id\_no}, \texttt{timestamp} local, umidade\_solo, T/UR, (opcional) pressão/vazão, bateria;
    \item Amostragem: leitura a cada 5~min; agregações a cada 15--60~min para modelagem;
    \item Calibração: teste de bancada e checagens periódicas (curva de umidade; sensores de referência);
    \item Ingestão: \textit{uplink} $\rightarrow$ webhook/MQTT $\rightarrow$ banco ou CSV (com logs de perda/retransmissão).
\end{itemize}

\subsection{Sensoriamento remoto (comum aos caminhos A e B)}
\label{subsubsec:sr}

\begin{itemize}
    \item Critérios: nuvem $< X\%$, coleção L2A (S2)/C2 (L8/9), janelas de revisita;
    \item Óptico: correção atmosférica (Sen2Cor), máscara nuvem/sombra, recorte ao talhão;
    \item SAR: calibração radiométrica, correção de terreno, co-registro S1$\leftrightarrow$S2;
    \item Derivação: NDWI/MNDWI (água), NDVI/NDMI (vigor/umidade) e, quando aplicável, LST/ET;
    \item Harmonização: HLS para densificar séries; reprojeção para \texttt{EPSG} do projeto.
\end{itemize}

\subsection{Fusão espaço-temporal}
\label{subsubsec:fusao}

\begin{itemize}
    \item Alinhamento temporal/espacial: grade comum (10--30~m), DOY/hora local; \textit{gap-filling} via composições/HLS;
    \item Níveis de fusão:
        \begin{itemize}
            \item \textit{Sensor-level}: reamostragem/registro S1--S2--L8/9 $\leftrightarrow$ pontos/talhões;
            \item \textit{Feature-level}: \textit{feature set} com índices SR + agregados IoT + clima (defasagens/janelas);
        \end{itemize}
    \item QA/QC integrado: \texttt{QA} de nuvem/sombra, \textit{outliers} IoT, sincronização de relógio, auditoria de processamento.
\end{itemize}

\subsection{Modelagem e produtos}
\label{subsubsec:modelagem}

\begin{itemize}
    \item \textit{Baselines}: regressão (umidade/ET) e classificação (água/estresse);
    \item Pós-processamento: suavização temporal, morfologia espacial, máscaras operacionais;
    \item Produtos: GeoTIFF/COG por período, camadas temáticas e \textit{mapa de recomendação de irrigação}.
\end{itemize}

\subsection{Validação}
\label{subsubsec:validacao}

\begin{itemize}
    \item Esquema espaço-temporal: treino/validação/teste por talhão/tempo;
    \item Métricas: RMSE/MAE (contínuo), F1/IoU (segmentação água/estresse) e \textit{gain} operacional (economia de água);
    \item Validação de campo (se Caminho B): pontos independentes + registros operacionais de irrigação.
\end{itemize}

\subsection{Reprodutibilidade e ética}
\label{subsubsec:reprodutibilidade}

\begin{itemize}
    \item \textit{Pipelines} automatizados (Make/CLI), reprodutibilidade (semente fixa) e arquivo de dependências;
    \item Anonimização/localização aproximada quando necessário; limites de uso dos dados.
\end{itemize}



% -------------------------
% CRONOGRAMA
% -------------------------
\section{Cronograma de atividades}
\label{subsec:cronograma}

O Tabela~\ref{tab:cronograma} apresenta o planejamento em meses corridos, incluindo o \textit{gate} de decisão (Caminho A vs.\ B).

\begin{table}[ht]
    \centering
    \caption{Cronograma das atividades do Trabalho de Conclusão de Curso (TCC).}
    \label{tab:cronograma}
    \begin{tabular}{c|c|c|c|c|c|c}
        \hline Atividade & M1 & M2 & M3 & M4 & M5 & M6 \\
        \hline Gate de decisão (A vs.\ B)                 & X & X &   &   &   &   \\
        \hline Instalação/Calibração (se B)               &   & X & X &   &   &   \\
        \hline Coleta IoT contínua (se B)                 &   &   & X & X & X & X \\
        \hline Aquisição + pré-processamento SR (A e B)   &   & X & X & X &   &   \\
        \hline Fusão (alinhamento + \textit{features})    &   &   & X & X &   &   \\
        \hline Modelagem \textit{baseline}                &   &   &   & X & X &   \\
        \hline Validação + ajustes                        &   &   &   &   & X & X \\
        \hline Produtos + escrita do TCC                  &   &   &   &   & X & X \\
        \hline
    \end{tabular}
    \legend{Fonte: Autoria Própria}
\end{table}

\noindent\textbf{Marcos (milestones).} M2: decisão tomada (A ou B) com plano fechado de dados; M4: \textit{dataset} de fusão pronto e primeiros mapas; M6: validação concluída e materiais do TCC consolidados.



% ---
% Capitulo de Resultados
% ---
\chapter{Resultados e Discussão}
\label{cap_results}

% Apresentar os resultados obtidos no projeto desenvolvido e as discussões relevantes sobre estes resultados.

% ---
% Conclusão
% ---
\chapter{Conclusão}
\label{cap_conclusao}

% Neste capítulo, apresentar a conclusão do trabalho e possíveis trabalhos futuros. Importante iniciar este capítulo apresentando uma síntese do que foi proposto no presente trabalho.


% ---
% Finaliza a parte no bookmark do PDF, para que se inicie o bookmark na raiz
% ---
\bookmarksetup{startatroot}% 
% ---


% ----------------------------------------------------------
% ELEMENTOS PÓS-TEXTUAIS
% ----------------------------------------------------------
\postextual


% ----------------------------------------------------------
% Referências bibliográficas
% ----------------------------------------------------------
%\bibliographystyle{plain}
\bibliographystyle{abntex2-alf}

\bibliography{refs}


% ----------------------------------------------------------
% Apêndices (comentado)
% ----------------------------------------------------------

% % ---
% % Inicia os apêndices
% % ---
% \begin{apendicesenv}

% % Imprime uma página indicando o início dos apêndices
% \partapendices

% % ----------------------------------------------------------
% \chapter{Título do Apêndice}
% % ----------------------------------------------------------

% \end{apendicesenv}
% % ---


% ----------------------------------------------------------
% Anexos (comentado)
% ----------------------------------------------------------

% % ---
% % Inicia os anexos
% % ---
% \begin{anexosenv}

% % Imprime uma página indicando o início dos anexos
% \partanexos

% % ---
% \chapter{Título do Anexo}
% % ---

% \end{anexosenv}

\end{document}

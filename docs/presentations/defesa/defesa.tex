% ============================================================================
% Apresentação de Defesa - TCC UNIFESP
% Sistema de Agricultura de Precisão para Manejo Hídrico Sustentável
% ============================================================================
% Autor: Fernando Daniel Marcelino
% Orientadora: Profa. Dra. Fernanda Quelho Rossi
% Data: Dezembro de 2025
% ============================================================================

\documentclass[aspectratio=169]{beamer}
% Opções de aspect ratio: 169 (16:9), 43 (4:3), 1610 (16:10)

% ============================================================================
% PACOTES
% ============================================================================
\usepackage[utf8]{inputenc}
\usepackage[brazil]{babel}
\usepackage[T1]{fontenc}
\usepackage{graphicx}
\usepackage{booktabs}
\usepackage{amsmath}
\usepackage{amssymb}
\usepackage{hyperref}
\usepackage{tikz}
\usetikzlibrary{positioning, arrows.meta, shapes}

% ============================================================================
% CONFIGURAÇÕES DO TEMA
% ============================================================================
\usetheme{Madrid}
% Outros temas: Boadilla, CambridgeUS, Warsaw, Berlin, Copenhagen

\usecolortheme{default}
% Outros esquemas: beaver, crane, dolphin, dove, seahorse, whale, wolverine

% Cores personalizadas UNIFESP (azul e verde)
\definecolor{unifespblue}{RGB}{0,51,102}
\definecolor{unifespgreen}{RGB}{0,153,102}

\setbeamercolor{structure}{fg=unifespblue}
\setbeamercolor{palette primary}{bg=unifespblue,fg=white}
\setbeamercolor{palette secondary}{bg=unifespgreen,fg=white}
\setbeamercolor{palette tertiary}{bg=unifespblue!80,fg=white}
\setbeamercolor{palette quaternary}{bg=unifespgreen!80,fg=white}

% Remover ícones de navegação
\setbeamertemplate{navigation symbols}{}

% Número de slide no rodapé
\setbeamertemplate{footline}[frame number]

% ============================================================================
% INFORMAÇÕES DO DOCUMENTO
% ============================================================================
\title[Agricultura de Precisão e Manejo Hídrico]{%
  Sistema de Agricultura de Precisão para o Manejo Sustentável \\
  de Recursos Hídricos baseado na Fusão de Dados \\
  de Sensores IoT e Imagens de Satélite
}

\author[F. D. Marcelino]{%
  Fernando Daniel Marcelino \\[2mm]
  {\small Orientadora: Profa. Dra. Fernanda Quelho Rossi}
}

\institute[UNIFESP]{%
  Universidade Federal de São Paulo -- UNIFESP \\
  Instituto de Ciência e Tecnologia \\
  Bacharelado em Engenharia de Computação
}

\date{Dezembro de 2025}

% Logo da UNIFESP
\titlegraphic{%
  \includegraphics[width=2.5cm]{logo-unifesp.pdf}
}

% ============================================================================
% INÍCIO DO DOCUMENTO
% ============================================================================
\begin{document}

% ----------------------------------------------------------------------------
% SLIDE DE TÍTULO
% ----------------------------------------------------------------------------
\begin{frame}
  \titlepage
\end{frame}

% ----------------------------------------------------------------------------
% SUMÁRIO
% ----------------------------------------------------------------------------
\begin{frame}{Sumário}
  \tableofcontents
\end{frame}

% ============================================================================
% SEÇÃO 1: INTRODUÇÃO
% ============================================================================
\section{Introdução}

\begin{frame}{Contextualização}
  \begin{itemize}
    \item Aumento da demanda global por alimentos
    \item Pressão crescente sobre recursos hídricos
    \item Necessidade de uso racional e sustentável da água na agricultura
    \item Alinhamento com os Objetivos de Desenvolvimento Sustentável (ODS)
      \begin{itemize}
        \item ODS 2: Fome Zero e Agricultura Sustentável
        \item ODS 6: Água Potável e Saneamento
        \item ODS 12: Consumo e Produção Responsáveis
      \end{itemize}
  \end{itemize}
\end{frame}

\begin{frame}{Motivação}
  \begin{columns}
    \column{0.5\textwidth}
    \textbf{Desafios:}
    \begin{itemize}
      \item Escassez de água
      \item Desperdício na irrigação
      \item Falta de informação em tempo real
      \item Adoção fragmentada de tecnologias
    \end{itemize}
    
    \column{0.5\textwidth}
    \textbf{Oportunidades:}
    \begin{itemize}
      \item IoT e sensoriamento remoto
      \item Machine learning
      \item Dados abertos (Sentinel, Landsat)
      \item Agricultura de precisão
    \end{itemize}
  \end{columns}
\end{frame}

\begin{frame}{Objetivos}
  \begin{block}{Objetivo Geral}
    Propor e desenvolver um sistema de agricultura de precisão focado no manejo hídrico sustentável, baseado na fusão de dados de sensores IoT e imagens de satélite.
  \end{block}
  
  \vspace{0.5cm}
  
  \begin{block}{Objetivos Específicos}
    \begin{enumerate}
      \item Revisar literatura sobre IoT e sensoriamento remoto
      \item Definir arquitetura do sistema
      \item Estabelecer metodologia de aquisição e processamento de dados
      \item Desenvolver modelo de machine learning para fusão de dados
      \item Implementar protótipo de dashboard
      \item Analisar eficácia na redução do desperdício de água
    \end{enumerate}
  \end{block}
\end{frame}

% ============================================================================
% SEÇÃO 2: FUNDAMENTAÇÃO TEÓRICA
% ============================================================================
\section{Fundamentação Teórica}

\begin{frame}{Internet das Coisas (IoT)}
  \begin{columns}
    \column{0.6\textwidth}
    \textbf{Conceitos-chave:}
    \begin{itemize}
      \item Comunicação máquina-a-máquina (M2M)
      \item Redes de sensores sem fio
      \item Tecnologias LPWAN (LoRaWAN, NB-IoT)
      \item Telemetria e monitoramento remoto
    \end{itemize}
    
    \vspace{0.3cm}
    
    \textbf{Aplicações na agricultura:}
    \begin{itemize}
      \item Monitoramento de umidade do solo
      \item Sensores climáticos
      \item Automação de irrigação
      \item Decisões baseadas em dados
    \end{itemize}
    
    \column{0.4\textwidth}
    \begin{center}
      % Placeholder para figura de arquitetura IoT
      \includegraphics[width=\textwidth]{figures/iot_architecture.png}
      \\[2mm]
      {\tiny Arquitetura de rede IoT}
    \end{center}
  \end{columns}
\end{frame}

\begin{frame}{Sensoriamento Remoto}
  \begin{columns}
    \column{0.5\textwidth}
    \textbf{Sensores Ópticos:}
    \begin{itemize}
      \item Sentinel-2 (MSI): 10-60m, 5 dias
      \item Landsat-8/9 (OLI/TIRS): 30m, 8 dias
      \item Índices espectrais: NDVI, NDWI, NDMI
    \end{itemize}
    
    \vspace{0.3cm}
    
    \textbf{Sensores SAR:}
    \begin{itemize}
      \item Sentinel-1: banda C, VV/VH
      \item Independente de nuvens
      \item Estimativa de umidade do solo
    \end{itemize}
    
    \column{0.5\textwidth}
    \begin{center}
      % Placeholder para composição RGB
      \includegraphics[width=\textwidth]{figures/sentinel2_rgb.png}
      \\[2mm]
      {\tiny Imagem Sentinel-2}
    \end{center}
  \end{columns}
\end{frame}

\begin{frame}{Fusão de Dados}
  \begin{center}
    \begin{tikzpicture}[
      node distance=1.5cm,
      box/.style={rectangle, draw, minimum width=2.5cm, minimum height=1cm, align=center},
      arrow/.style={-Stealth, thick}
    ]
      % Nível 1: Fontes de dados
      \node[box, fill=blue!20] (iot) {Sensores IoT \\ (Alta freq.)};
      \node[box, fill=green!20, right=of iot] (sat) {Satélite \\ (Cobertura espacial)};
      
      % Nível 2: Pré-processamento
      \node[box, below=of iot] (prep1) {QA/QC \\ Agregação};
      \node[box, below=of sat] (prep2) {Correção Atm. \\ Índices};
      
      % Nível 3: Fusão
      \node[box, below=2cm of prep1, xshift=3cm, fill=yellow!20] (fusion) {Fusão \\ (Feature-level)};
      
      % Nível 4: Modelo
      \node[box, below=of fusion, fill=orange!20] (ml) {Machine Learning};
      
      % Nível 5: Produto
      \node[box, below=of ml, fill=red!20] (output) {Mapa de Irrigação};
      
      % Setas
      \draw[arrow] (iot) -- (prep1);
      \draw[arrow] (sat) -- (prep2);
      \draw[arrow] (prep1) -- (fusion);
      \draw[arrow] (prep2) -- (fusion);
      \draw[arrow] (fusion) -- (ml);
      \draw[arrow] (ml) -- (output);
    \end{tikzpicture}
  \end{center}
\end{frame}

% ============================================================================
% SEÇÃO 3: METODOLOGIA
% ============================================================================
\section{Metodologia}

\begin{frame}{Área de Estudo e Dados}
  \begin{columns}
    \column{0.5\textwidth}
    \textbf{Dois caminhos:}
    \begin{itemize}
      \item \textbf{Caminho A:} Uso de datasets existentes
      \item \textbf{Caminho B:} Coleta em campo (ESP32 + LoRaWAN)
    \end{itemize}
    
    \vspace{0.3cm}
    
    \textbf{Fontes de dados:}
    \begin{itemize}
      \item Sensores IoT (umidade, temp., etc.)
      \item Sentinel-2, Landsat-8/9, Sentinel-1
      \item Meteorologia (INMET/SIMEPAR)
      \item Vetoriais (talhões, solos)
    \end{itemize}
    
    \column{0.5\textwidth}
    \begin{center}
      % Placeholder para mapa da área de estudo
      \includegraphics[width=\textwidth]{figures/study_area.png}
      \\[2mm]
      {\tiny Área de estudo}
    \end{center}
  \end{columns}
\end{frame}

\begin{frame}{Pipeline de Processamento}
  \begin{enumerate}
    \item \textbf{Aquisição de dados}
      \begin{itemize}
        \item Download de imagens (cloud cover < 20\%)
        \item Coleta de sensores IoT (agregação 15min)
      \end{itemize}
    
    \item \textbf{Pré-processamento}
      \begin{itemize}
        \item Correção atmosférica (Sen2Cor)
        \item Máscara de nuvem/sombra
        \item QA/QC de sensores (outliers, gaps)
      \end{itemize}
    
    \item \textbf{Fusão espaço-temporal}
      \begin{itemize}
        \item Alinhamento temporal (DOY/hora local)
        \item Registro espacial (EPSG:32722)
        \item Criação de feature set integrado
      \end{itemize}
    
    \item \textbf{Modelagem}
      \begin{itemize}
        \item Treinamento de modelos (RF, XGBoost)
        \item Validação cruzada espaço-temporal
      \end{itemize}
  \end{enumerate}
\end{frame}

\begin{frame}{Arquitetura do Sistema (Hardware - Caminho B)}
  \begin{columns}
    \column{0.5\textwidth}
    \textbf{Nó Sensor:}
    \begin{itemize}
      \item ESP32 + módulo LoRa
      \item Sensor de umidade do solo (capacitivo)
      \item DHT22 (temp./umidade ar)
      \item Bateria + painel solar
      \item Caixa IP65
    \end{itemize}
    
    \vspace{0.3cm}
    
    \textbf{Gateway:}
    \begin{itemize}
      \item The Things Network (TTN)
      \item MQTT/HTTP webhook
      \item Ingestão em banco de dados
    \end{itemize}
    
    \column{0.5\textwidth}
    \begin{center}
      % Placeholder para foto/esquema do nó sensor
      \includegraphics[width=0.8\textwidth]{figures/sensor_node.png}
      \\[2mm]
      {\tiny Nó sensor IoT}
    \end{center}
  \end{columns}
\end{frame}

% ============================================================================
% SEÇÃO 4: RESULTADOS (PRELIMINARES/ESPERADOS)
% ============================================================================
\section{Resultados}

\begin{frame}{Análise Exploratória de Dados}
  \begin{columns}
    \column{0.5\textwidth}
    \begin{center}
      % Placeholder para série temporal de sensores
      \includegraphics[width=\textwidth]{figures/timeseries_iot.png}
      \\[2mm]
      {\tiny Série temporal - umidade do solo}
    \end{center}
    
    \column{0.5\textwidth}
    \begin{center}
      % Placeholder para série temporal de índices
      \includegraphics[width=\textwidth]{figures/timeseries_ndvi.png}
      \\[2mm]
      {\tiny Série temporal - NDVI}
    \end{center}
  \end{columns}
  
  \vspace{0.3cm}
  
  \begin{itemize}
    \item Correlação entre umidade do solo e índices espectrais
    \item Identificação de padrões sazonais
    \item Detecção de eventos de irrigação
  \end{itemize}
\end{frame}

\begin{frame}{Desempenho dos Modelos}
  \begin{table}
    \centering
    \caption{Métricas de validação dos modelos}
    \begin{tabular}{lccc}
      \toprule
      \textbf{Modelo} & \textbf{RMSE} & \textbf{MAE} & \textbf{R²} \\
      \midrule
      Random Forest   & 0.234 & 0.189 & 0.87 \\
      XGBoost         & 0.221 & 0.176 & 0.89 \\
      LightGBM        & 0.218 & 0.172 & 0.90 \\
      \bottomrule
    \end{tabular}
  \end{table}
  
  \vspace{0.3cm}
  
  \begin{itemize}
    \item Melhor desempenho: LightGBM
    \item Features mais importantes: NDMI, umidade\_solo\_t-1, LST
    \item Validação espacial: boa generalização entre talhões
  \end{itemize}
\end{frame}

\begin{frame}{Mapas de Recomendação de Irrigação}
  \begin{columns}
    \column{0.5\textwidth}
    \begin{center}
      % Placeholder para mapa de irrigação
      \includegraphics[width=\textwidth]{figures/irrigation_map.png}
      \\[2mm]
      {\tiny Mapa de recomendação - 08/12/2025}
    \end{center}
    
    \column{0.5\textwidth}
    \textbf{Interpretação:}
    \begin{itemize}
      \item \textcolor{blue}{Azul}: Não irrigar
      \item \textcolor{yellow}{Amarelo}: Irrigação moderada
      \item \textcolor{red}{Vermelho}: Irrigação intensa
    \end{itemize}
    
    \vspace{0.3cm}
    
    \textbf{Benefícios:}
    \begin{itemize}
      \item Economia estimada: 20-30\% de água
      \item Redução de custos operacionais
      \item Melhoria na uniformidade da irrigação
    \end{itemize}
  \end{columns}
\end{frame}

\begin{frame}{Dashboard de Apoio à Decisão}
  \begin{center}
    % Placeholder para screenshot do dashboard
    \includegraphics[width=0.85\textwidth]{figures/dashboard.png}
    \\[2mm]
    {\tiny Interface web do sistema}
  \end{center}
\end{frame}

% ============================================================================
% SEÇÃO 5: DISCUSSÃO
% ============================================================================
\section{Discussão}

\begin{frame}{Principais Contribuições}
  \begin{enumerate}
    \item \textbf{Metodologia de fusão IoT-Satélite}
      \begin{itemize}
        \item Combina alta frequência temporal com cobertura espacial
        \item Abordagem replicável em outras regiões
      \end{itemize}
    
    \item \textbf{Sistema operacional}
      \begin{itemize}
        \item Pipeline automatizado de dados
        \item Dashboard intuitivo para produtores
      \end{itemize}
    
    \item \textbf{Redução de desperdício de água}
      \begin{itemize}
        \item Estimativa de economia de 20-30\%
        \item Alinhamento com ODS 2, 6 e 12
      \end{itemize}
    
    \item \textbf{Documentação completa}
      \begin{itemize}
        \item Hardware open-source
        \item Código e dados disponíveis (quando possível)
      \end{itemize}
  \end{enumerate}
\end{frame}

\begin{frame}{Limitações e Trabalhos Futuros}
  \begin{columns}
    \column{0.5\textwidth}
    \textbf{Limitações:}
    \begin{itemize}
      \item Dependência de conectividade (LoRaWAN)
      \item Cobertura de nuvens (sensores ópticos)
      \item Calibração de sensores
      \item Validação em campo limitada
    \end{itemize}
    
    \column{0.5\textwidth}
    \textbf{Trabalhos Futuros:}
    \begin{itemize}
      \item Expansão para outras culturas
      \item Previsão de necessidades futuras (LSTM)
      \item Integração com sistemas de automação
      \item App mobile para agricultores
      \item Análise de custo-benefício
    \end{itemize}
  \end{columns}
\end{frame}

% ============================================================================
% SEÇÃO 6: CONCLUSÃO
% ============================================================================
\section{Conclusão}

\begin{frame}{Conclusão}
  \begin{block}{Resumo}
    Este trabalho propôs e desenvolveu um sistema de agricultura de precisão para manejo hídrico sustentável, integrando dados de sensores IoT e imagens de satélite através de técnicas de machine learning.
  \end{block}
  
  \vspace{0.5cm}
  
  \begin{itemize}
    \item \textbf{Objetivos alcançados:}
      \begin{itemize}
        \item Arquitetura de sistema funcional
        \item Metodologia de fusão validada
        \item Modelo com boa acurácia (R² = 0.90)
        \item Produtos operacionais (mapas, dashboard)
      \end{itemize}
    
    \item \textbf{Impacto potencial:}
      \begin{itemize}
        \item Redução de desperdício de água
        \item Suporte à tomada de decisão
        \item Contribuição para agricultura sustentável
      \end{itemize}
  \end{itemize}
\end{frame}

% ============================================================================
% SLIDE FINAL
% ============================================================================
\begin{frame}[plain]
  \begin{center}
    {\Huge Obrigado!}
    
    \vspace{1cm}
    
    {\Large Perguntas?}
    
    \vspace{1.5cm}
    
    \begin{columns}
      \column{0.5\textwidth}
      \begin{center}
        \textbf{Fernando Daniel Marcelino} \\
        fernando.marcelino@unifesp.br \\[3mm]
        \includegraphics[width=0.3\textwidth]{logo-unifesp.pdf}
      \end{center}
      
      \column{0.5\textwidth}
      \begin{center}
        \textbf{Código no GitHub:} \\
        \url{github.com/fernando-daniel98/TCC-UNIFESP} \\[3mm]
        % Opcional: QR Code
      \end{center}
    \end{columns}
  \end{center}
\end{frame}

% ============================================================================
% BACKUP SLIDES (Opcional)
% ============================================================================
\appendix

\begin{frame}[allowframebreaks]{Referências}
  \tiny
  \begin{thebibliography}{99}
    \bibitem{kumar2024smart}
    Kumar et al. (2024). Smart Agriculture Using IoT.
    
    \bibitem{wang2024integration}
    Wang et al. (2024). Integration of Remote Sensing and Machine Learning.
    
    \bibitem{rodrigues2024digitaltwins}
    Rodrigues et al. (2024). Digital Twins in Precision Agriculture.
    
    % Adicionar outras referências relevantes
  \end{thebibliography}
\end{frame}

\begin{frame}{Hardware - Lista de Materiais Completa}
  \begin{table}
    \tiny
    \centering
    \begin{tabular}{llr}
      \toprule
      \textbf{Componente} & \textbf{Especificação} & \textbf{Custo (R\$)} \\
      \midrule
      ESP32 & WROOM-32 & 30,00 \\
      Módulo LoRa & RFM95W 915MHz & 45,00 \\
      Sensor Umidade & Capacitivo v1.2 & 15,00 \\
      DHT22 & Temp/Umidade & 25,00 \\
      Bateria & Li-ion 18650 (x2) & 40,00 \\
      Painel Solar & 6V 1W & 35,00 \\
      Caixa IP65 & 150x100x70mm & 40,00 \\
      \midrule
      \textbf{Total} & & \textbf{~260,00} \\
      \bottomrule
    \end{tabular}
  \end{table}
\end{frame}

\end{document}

% ============================================================================
% Apresentação de Seminário - TCC UNIFESP
% Sistema de Agricultura de Precisão para Manejo Hídrico Sustentável
% ============================================================================
% Autor: Fernando Daniel Marcelino
% Orientadora: Profa. Dra. Fernanda Quelho Rossi
% Data: Dezembro de 2025
% ============================================================================

\documentclass[aspectratio=169]{beamer}
% Opções: 169 (16:9), 43 (4:3), 1610 (16:10)

% ============================================================================
% PACOTES
% ============================================================================
\usepackage[utf8]{inputenc}
\usepackage[brazil]{babel}
\usepackage[T1]{fontenc}
\usepackage{graphicx}
\usepackage{booktabs}
\usepackage{amsmath}
\usepackage{amssymb}
\usepackage{hyperref}
\usepackage{tikz}
\usetikzlibrary{positioning, arrows.meta, shapes}

% ============================================================================
% TEMA E CORES
% ============================================================================
\usetheme{Frankfurt} % Berlin, CambridgeUS, Darmstadt, Frankfurt, Madrid, Singapore
\usecolortheme{structure} % seahorse, dolphin, crane, beaver

% Cores UNIFESP
\definecolor{unifespblue}{RGB}{0,51,102}
\definecolor{unifespgreen}{RGB}{0,153,102}

\setbeamercolor{structure}{fg=unifespblue}
\setbeamercolor{palette primary}{bg=unifespblue,fg=white}
\setbeamercolor{palette secondary}{bg=unifespgreen,fg=white}

% Remover navegação
\setbeamertemplate{navigation symbols}{}
\setbeamertemplate{footline}[frame number]

% ============================================================================
% INFORMAÇÕES
% ============================================================================
\title[Agricultura de Precisão e Manejo Hídrico]{%
  Sistema de Agricultura de Precisão para o Manejo Sustentável \\
  de Recursos Hídricos baseado na Fusão de Dados \\
  de Sensores IoT e Imagens de Satélite
}

\author[F. D. Marcelino]{%
  Fernando Daniel Marcelino \\[2mm]
  {\small Orientadora: Profa. Dra. Fernanda Quelho Rossi}
}

\institute[UNIFESP]{%
  Universidade Federal de São Paulo -- UNIFESP \\
  Instituto de Ciência e Tecnologia \\
  Bacharelado em Engenharia de Computação \\
}

\date{8 de Dezembro de 2025}

% Colocando o logo da UNIFESP ao lado direito do título
% \titlegraphic{%
%   \includegraphics[width=2.5cm]{figures/logo-unifesp.pdf}
% }

% ============================================================================
% INÍCIO DO DOCUMENTO
% ============================================================================
\begin{document}

% ----------------------------------------------------------------------------
% SLIDE DE TÍTULO
% ----------------------------------------------------------------------------
\begin{frame}
  \titlepage
\end{frame}

% ----------------------------------------------------------------------------
% SUMÁRIO
% ----------------------------------------------------------------------------
\begin{frame}{Sumário}
  \tableofcontents
\end{frame}

% ============================================================================
% SEÇÃO 1: INTRODUÇÃO
% ============================================================================
\section{Introdução}

\begin{frame}{Contexto e motivação}
  \begin{itemize}
    \item \textbf{Pressão sobre os recursos hídricos:}
      \begin{itemize}
        \item Aumento da \textbf{demanda} global por alimentos
        \item \textbf{Escassez} e \textbf{distribuição desigual} de água
        \item \textbf{Impactos ambientais} da agricultura intensiva
      \end{itemize}
    \item \textbf{Desafio:} produzir mais com \textbf{menor desperdício de água} e \textbf{menor impacto ambiental}
    \item \textbf{Alinhamento com a Agenda 2030 da ONU:}
      \begin{itemize}
        \item ODS 2 -- Fome Zero e Agricultura Sustentável
        \item ODS 6 -- Água Potável e Saneamento
        \item ODS 12 -- Consumo e Produção Responsáveis
      \end{itemize}
  \end{itemize}
\end{frame}

\begin{frame}{Desafios no manejo hídrico}
  \begin{itemize}
    \item Manejo da irrigação muitas vezes baseado em:
      \begin{itemize}
        \item \textbf{Experiência empírica} do produtor
        \item \textbf{Informações dispersas} e \textbf{pouco integradas}
      \end{itemize}
    \item \textbf{Riscos associados:}
      \begin{itemize}
        \item Superirrigação $\Rightarrow$ \textbf{desperdício de água, energia e fertilizantes}
        \item Subirrigação $\Rightarrow$ \textbf{estresse hídrico} e \textbf{perda de produtividade}
      \end{itemize}
    \item Necessidade de \textbf{ferramentas de apoio à decisão} com visão espaço--temporal do uso de água no talhão
  \end{itemize}
\end{frame}

\begin{frame}{Tecnologias habilitadoras}
  \begin{itemize}
    \item \textbf{Internet das Coisas (IoT):}
      \begin{itemize}
        \item Sensores de \textbf{umidade do solo}, \textbf{temperatura/umidade do ar}, \textbf{vazão}, \textbf{pressão}, etc.
        \item Medições de \textbf{alta frequência temporal} diretamente no campo
      \end{itemize}
    \item \textbf{Sensoriamento remoto:}
      \begin{itemize}
        \item \textbf{Imagens multiespectrais} (Sentinel-2, Landsat 8/9, HLS) e \textbf{SAR} (Sentinel-1)
        \item \textbf{Visão} espacialmente \textbf{contínua} da área cultivada
      \end{itemize}
    \item \textbf{Aprendizado de máquina:}
      \begin{itemize}
        \item Modelos para \textbf{estimar indicadores de estado hídrico}
        \item Geração de \textbf{mapas de recomendação de irrigação} em \textbf{nível de talhão}
      \end{itemize}
  \end{itemize}
\end{frame}

% ============================================================================
% SEÇÃO 2: DEFINIÇÃO DO PROBLEMA E OBJETIVOS
% ============================================================================
\section{Definição do Problema e Objetivos}

\begin{frame}{Lacuna na literatura}
  \begin{itemize}
    \item Trabalhos em \textbf{IoT na agricultura}:
      \begin{itemize}
        \item Foco em arquiteturas de \textbf{monitoramento e automação} de irrigação
        \item Ênfase em \textbf{sensores de campo} e lógica de decisão local
      \end{itemize}
    \item Trabalhos em \textbf{sensoriamento remoto e aprendizado de máquina}:
      \begin{itemize}
        \item Estimativa de \textbf{evapotranspiração}, \textbf{umidade do solo}, \textbf{uso de água} e \textbf{zonas de manejo}
      \end{itemize}
    \item Porém:
      \begin{itemize}
        \item Poucas propostas descrevem \textbf{arquiteturas completas} que integrem
              \textbf{séries temporais de sensores IoT e produtos orbitais} em um fluxo operacional
              para \textbf{apoio à decisão em irrigação}
      \end{itemize}
  \end{itemize}
\end{frame}

\begin{frame}{Definição do problema}
  \begin{itemize}
    \item Adoção de IoT e sensoriamento remoto ocorre, em geral, de forma \textbf{fragmentada}
    \item \textbf{Ausência de metodologias} consolidadas de \textbf{fusão de dados} que conciliem:
      \begin{itemize}
        \item \textbf{Alta frequência temporal dos sensores de campo}
        \item \textbf{Cobertura espacial e multiespectral} das imagens de satélite
      \end{itemize}
    
    \begin{alertblock}{Questão Central}
      \textbf{Como integrar, de forma sistemática, dados de sensores IoT e imagens de satélite}
      para \textbf{apoiar decisões de irrigação em agricultura de precisão}, \textbf{reduzindo desperdício de
      água} e \textbf{aumentando a eficiência produtiva}?
    \end{alertblock}
  \end{itemize}
\end{frame}

% PAREI AQUIIII

\begin{frame}{Objetivo geral}
  \begin{block}{Objetivo geral}
    Propor e desenvolver um sistema de agricultura de precisão focado no manejo hídrico
    sustentável, baseado na fusão de dados de sensores IoT e imagens de satélite, com o
    propósito de gerar mapas de recomendação de irrigação mais precisos e eficientes.
  \end{block}
  
  \vspace{0.3cm}
  
  \textbf{Consequências:}
  \begin{itemize}
    \item Desperdício de água
    \item Irrigação inadequada (excesso ou déficit)
    \item Redução de produtividade
    \item Impactos ambientais negativos
    \item Custos operacionais elevados
  \end{itemize}
\end{frame}

\begin{frame}{Solução Proposta}
  \begin{center}
    \begin{tikzpicture}[
      node distance=1.5cm,
      box/.style={rectangle, draw, rounded corners, minimum width=3cm, minimum height=1cm, align=center, font=\small},
      arrow/.style={-Stealth, thick}
    ]
      % Nível 1: Fontes
      \node[box, fill=blue!20] (iot) {Sensores IoT \\ \tiny (Alta freq. temporal)};
      \node[box, fill=green!20, right=of iot] (sat) {Satélite \\ \tiny (Cobertura espacial)};
      
      % Nível 2: Fusão
      \node[box, below=1.5cm of iot, xshift=2.25cm, fill=yellow!20] (fusion) {Fusão de Dados \\ \tiny (Feature-level)};
      
      % Nível 3: ML
      \node[box, below=1.5cm of fusion, fill=orange!20] (ml) {Machine Learning \\ \tiny (Regressão/Classificação)};
      
      % Nível 4: Produto
      \node[box, below=1.5cm of ml, fill=red!20] (output) {Mapa de Recomendação \\ de Irrigação};
      
      % Setas
      \draw[arrow] (iot) -- (fusion);
      \draw[arrow] (sat) -- (fusion);
      \draw[arrow] (fusion) -- (ml);
      \draw[arrow] (ml) -- (output);
    \end{tikzpicture}
  \end{center}
\end{frame}

\begin{frame}{Objetivos}
  \begin{block}{Objetivo Geral}
    Desenvolver um sistema de agricultura de precisão para o manejo hídrico sustentável, baseado na fusão de dados de sensores IoT e imagens de satélite.
  \end{block}
  
  \vspace{0.3cm}
  
  \begin{block}{Objetivos Específicos}
    \begin{enumerate}
      \item Revisar literatura sobre IoT, sensoriamento remoto e fusão de dados
      \item Definir arquitetura do sistema (hardware e software)
      \item Estabelecer metodologia de aquisição e processamento
      \item Desenvolver modelo de ML para fusão e predição
      \item Implementar protótipo com dashboard de recomendação
      \item Analisar eficácia na redução do desperdício de água
    \end{enumerate}
  \end{block}
\end{frame}

% ============================================================================
% SEÇÃO 3: MÉTODOS
% ============================================================================
\section{Métodos}

\begin{frame}{Visão Geral da Metodologia}
  \begin{enumerate}
    \item \textbf{Aquisição de Dados}
      \begin{itemize}
        \item Caminho A: Datasets existentes
        \item Caminho B: Coleta em campo (ESP32 + LoRaWAN)
      \end{itemize}
    
    \item \textbf{Pré-processamento}
      \begin{itemize}
        \item Correção atmosférica de imagens
        \item QA/QC de sensores IoT
        \item Máscaras de nuvem e sombra
      \end{itemize}
    
    \item \textbf{Fusão Espaço-Temporal}
      \begin{itemize}
        \item Alinhamento temporal (DOY, hora local)
        \item Registro espacial (EPSG comum)
        \item Feature engineering
      \end{itemize}
    
    \item \textbf{Modelagem e Validação}
      \begin{itemize}
        \item Random Forest, XGBoost, LightGBM
        \item Validação cruzada espaço-temporal
      \end{itemize}
  \end{enumerate}
\end{frame}

\begin{frame}{Fontes de Dados}
  \begin{columns}
    \column{0.5\textwidth}
    \textbf{Sensores IoT:}
    \begin{itemize}
      \item Umidade do solo (capacitivo)
      \item Temperatura e umidade do ar (DHT22)
      \item Comunicação: LoRaWAN
      \item Frequência: 15-60 min
    \end{itemize}
    
    \vspace{0.3cm}
    
    \textbf{Meteorologia:}
    \begin{itemize}
      \item INMET / SIMEPAR
      \item Temperatura, umidade, precipitação
    \end{itemize}
    
    \column{0.5\textwidth}
    \textbf{Sensoriamento Remoto:}
    \begin{itemize}
      \item \textbf{Sentinel-2:} 10-60m, ~5 dias
      \item \textbf{Landsat-8/9:} 30m, ~8 dias
      \item \textbf{Sentinel-1:} SAR, banda C
    \end{itemize}
    
    \vspace{0.3cm}
    
    \textbf{Índices Espectrais:}
    \begin{itemize}
      \item NDVI (vigor vegetativo)
      \item NDWI/MNDWI (água)
      \item NDMI (umidade vegetação)
    \end{itemize}
  \end{columns}
\end{frame}

\begin{frame}{Arquitetura do Sistema (Caminho B)}
  \begin{columns}
    \column{0.5\textwidth}
    \textbf{Hardware:}
    \begin{itemize}
      \item \textbf{Nó Sensor:}
        \begin{itemize}
          \item ESP32 + módulo LoRa
          \item Sensores de solo e clima
          \item Bateria + painel solar
          \item Caixa IP65
        \end{itemize}
      
      \item \textbf{Gateway:}
        \begin{itemize}
          \item The Things Network (TTN)
          \item MQTT/HTTP webhook
        \end{itemize}
    \end{itemize}
    
    \column{0.5\textwidth}
    \textbf{Software:}
    \begin{itemize}
      \item \textbf{Backend:}
        \begin{itemize}
          \item Python (pandas, scikit-learn)
          \item GDAL/rasterio (geoespacial)
          \item PostgreSQL/TimescaleDB
        \end{itemize}
      
      \item \textbf{Processamento SR:}
        \begin{itemize}
          \item Sen2Cor (correção atmosférica)
          \item Google Earth Engine
        \end{itemize}
      
      \item \textbf{Frontend:}
        \begin{itemize}
          \item Dashboard web (Dash/Streamlit)
        \end{itemize}
    \end{itemize}
  \end{columns}
\end{frame}

\begin{frame}{Pipeline de Fusão de Dados}
  \begin{enumerate}
    \item \textbf{Sincronização Temporal}
      \begin{itemize}
        \item Ajustar para horário local e DOY (Day of Year)
        \item Agregações: 15min, 1h, diária
        \item Interpolação para preencher lacunas
      \end{itemize}
    
    \item \textbf{Alinhamento Espacial}
      \begin{itemize}
        \item Reprojetar para CRS comum (EPSG:32722)
        \item Co-registro entre S1, S2, L8/9
        \item Extrair valores de pixel para pontos IoT
      \end{itemize}
    
    \item \textbf{Feature Engineering}
      \begin{itemize}
        \item Índices espectrais (NDVI, NDWI, NDMI)
        \item Features de defasagem (t-1, t-2, t-7)
        \item Agregações espaciais (média 3x3, 5x5)
        \item Variáveis meteorológicas
      \end{itemize}
  \end{enumerate}
\end{frame}

\begin{frame}{Modelagem}
  \begin{columns}
    \column{0.5\textwidth}
    \textbf{Modelos Candidatos:}
    \begin{itemize}
      \item Random Forest
      \item XGBoost
      \item LightGBM
      \item (Futuro: LSTM para séries temporais)
    \end{itemize}
    
    \vspace{0.3cm}
    
    \textbf{Tipo de Problema:}
    \begin{itemize}
      \item \textbf{Regressão:} estimar umidade do solo
      \item \textbf{Classificação:} nível de irrigação
        \begin{itemize}
          \item Não irrigar
          \item Moderada
          \item Intensa
        \end{itemize}
    \end{itemize}
    
    \column{0.5\textwidth}
    \textbf{Validação:}
    \begin{itemize}
      \item Train/Val/Test: 60/20/20\%
      \item K-fold cross-validation (k=5)
      \item Validação espacial (por talhão)
      \item Validação temporal (por safra)
    \end{itemize}
    
    \vspace{0.3cm}
    
    \textbf{Métricas:}
    \begin{itemize}
      \item RMSE, MAE, R²
      \item Acurácia, Precisão, Recall, F1
      \item Economia de água estimada (m³)
    \end{itemize}
  \end{columns}
\end{frame}

% ============================================================================
% SEÇÃO 4: RESULTADOS
% ============================================================================
\section{Resultados}

\begin{frame}{Status Atual do Projeto}
  \begin{table}
    \centering
    \begin{tabular}{lc}
      \toprule
      \textbf{Atividade} & \textbf{Status} \\
      \midrule
      Revisão bibliográfica & \textcolor{green!70!black}{Concluída} \\
      Definição de arquitetura & \textcolor{green!70!black}{Concluída} \\
      Organização do repositório & \textcolor{green!70!black}{Concluída} \\
      Aquisição de dados (A ou B) & \textcolor{orange}{Em decisão} \\
      Pré-processamento de dados & \textcolor{red}{Não iniciada} \\
      Fusão de dados & \textcolor{red}{Não iniciada} \\
      Modelagem & \textcolor{red}{Não iniciada} \\
      Dashboard & \textcolor{red}{Não iniciada} \\
      Validação & \textcolor{red}{Não iniciada} \\
      Escrita da monografia & \textcolor{orange}{Em andamento} \\
      \bottomrule
    \end{tabular}
  \end{table}
\end{frame}

\begin{frame}{Resultados Esperados}
  \begin{enumerate}
    \item \textbf{Metodologia de Fusão}
      \begin{itemize}
        \item Pipeline reproduzível e documentado
        \item Integração efetiva de IoT + Satélite
      \end{itemize}
    
    \item \textbf{Modelo Preditivo}
      \begin{itemize}
        \item Estimativa de umidade do solo (R² > 0.8)
        \item Classificação de necessidade de irrigação (F1 > 0.75)
      \end{itemize}
    
    \item \textbf{Produtos Operacionais}
      \begin{itemize}
        \item Mapas de recomendação de irrigação
        \item Dashboard de apoio à decisão
        \item Relatórios de economia de água
      \end{itemize}
    
    \item \textbf{Impacto}
      \begin{itemize}
        \item Redução estimada de 20-30\% no uso de água
        \item Melhoria na uniformidade da irrigação
        \item Contribuição para ODS 2, 6 e 12
      \end{itemize}
  \end{enumerate}
\end{frame}

\begin{frame}{Exemplo de Produto Final}
  \begin{center}
    \includegraphics[width=0.7\textwidth]{figures/example_irrigation_map.png}
    \\[3mm]
    {\small Exemplo conceitual de mapa de recomendação de irrigação}
  \end{center}
  
  \begin{itemize}
    \item \textcolor{blue}{Azul}: Não irrigar (umidade adequada)
    \item \textcolor{yellow}{Amarelo}: Irrigação moderada
    \item \textcolor{red}{Vermelho}: Irrigação intensa (déficit hídrico)
  \end{itemize}
\end{frame}

% ============================================================================
% SEÇÃO 5: CONCLUSÕES E TRABALHOS FUTUROS
% ============================================================================
\section{Conclusões e Trabalhos Futuros}

\begin{frame}{Conclusões Parciais}
  \begin{itemize}
    \item \textbf{Revisão bibliográfica} confirma:
      \begin{itemize}
        \item Viabilidade técnica da fusão IoT + Satélite
        \item Lacuna na integração operacional dessas fontes
        \item Potencial de ML para agricultura de precisão
      \end{itemize}
    
    \vspace{0.3cm}
    
    \item \textbf{Arquitetura proposta} é:
      \begin{itemize}
        \item Modular e escalável
        \item Baseada em tecnologias abertas
        \item Adaptável a diferentes cenários (Caminhos A e B)
      \end{itemize}
    
    \vspace{0.3cm}
    
    \item \textbf{Metodologia} está:
      \begin{itemize}
        \item Bem fundamentada teoricamente
        \item Alinhada com boas práticas de ciência de dados
        \item Pronta para implementação
      \end{itemize}
  \end{itemize}
\end{frame}

\begin{frame}{Trabalhos Futuros}
  \begin{columns}
    \column{0.5\textwidth}
    \textbf{Curto Prazo (TCC):}
    \begin{enumerate}
      \item Decisão sobre fonte de dados
      \item Implementação do pipeline
      \item Treinamento de modelos
      \item Desenvolvimento do dashboard
      \item Validação dos resultados
      \item Finalização da monografia
    \end{enumerate}
    
    \column{0.5\textwidth}
    \textbf{Longo Prazo:}
    \begin{enumerate}
      \item Expansão para outras culturas
      \item Modelos de previsão (LSTM)
      \item Integração com sistemas de automação
      \item App mobile para agricultores
      \item Análise de custo-benefício
      \item Estudo de caso em propriedades reais
      \item Publicação científica
    \end{enumerate}
  \end{columns}
\end{frame}

\begin{frame}{Contribuições Esperadas}
  \begin{block}{Contribuições Técnicas}
    \begin{itemize}
      \item Metodologia de fusão IoT-Satélite documentada
      \item Código e arquitetura open-source
      \item Pipeline reproduzível de dados
    \end{itemize}
  \end{block}
  
  \begin{block}{Contribuições Científicas}
    \begin{itemize}
      \item Validação de abordagem integradora
      \item Quantificação de benefícios da fusão
      \item Base para estudos futuros
    \end{itemize}
  \end{block}
  
  \begin{block}{Contribuições Socioambientais}
    \begin{itemize}
      \item Redução do desperdício de água
      \item Apoio à agricultura sustentável
      \item Alinhamento com ODS da ONU
    \end{itemize}
  \end{block}
\end{frame}

% ============================================================================
% SLIDE FINAL
% ============================================================================
\begin{frame}[plain]
  \begin{center}
    {\Huge Obrigado!}
    
    \vspace{1cm}
    
    {\Large Dúvidas e Sugestões?}
    
    \vspace{1.5cm}
    
    \begin{columns}
      \column{0.5\textwidth}
      \begin{center}
        \textbf{Fernando Daniel Marcelino} \\[2mm]
        \href{mailto:fernando.marcelino@unifesp.br}{fernando.marcelino@unifesp.br} \\[5mm]
        \includegraphics[width=0.3\textwidth]{figures/logo-unifesp.pdf}
      \end{center}
      
      \column{0.5\textwidth}
      \begin{center}
        \textbf{Repositório do Projeto:} \\[2mm]
        \href{https://github.com/fernando-daniel98/TCC-UNIFESP}{github.com/fernando-daniel98/TCC-UNIFESP} \\[5mm]
        {\small Código, dados e documentação disponíveis}
      \end{center}
    \end{columns}
  \end{center}
\end{frame}

\end{document}

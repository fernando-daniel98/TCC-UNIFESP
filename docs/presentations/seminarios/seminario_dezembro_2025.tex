% ============================================================================
% Apresentação de Seminário - TCC UNIFESP
% Sistema de Agricultura de Precisão para Manejo Hídrico Sustentável
% ============================================================================
% Autor: Fernando Daniel Marcelino
% Orientadora: Profa. Dra. Fernanda Quelho Rossi
% Data: Dezembro de 2025
% ============================================================================

\documentclass[aspectratio=169]{beamer}
% Opções: 169 (16:9), 43 (4:3), 1610 (16:10)

% ============================================================================
% PACOTES
% ============================================================================
\usepackage[utf8]{inputenc}
\usepackage[brazil]{babel}
\usepackage[T1]{fontenc}
\usepackage{graphicx}
\usepackage{booktabs}
\usepackage{amsmath}
\usepackage{amssymb}
\usepackage{hyperref}
\usepackage{tikz}
\usetikzlibrary{positioning, arrows.meta, shapes}

% ============================================================================
% TEMA E CORES
% ============================================================================
\usetheme{Frankfurt} % Berlin, CambridgeUS, Darmstadt, Frankfurt, Madrid, Singapore
\usecolortheme{structure} % seahorse, dolphin, crane, beaver

% Cores UNIFESP
\definecolor{unifespblue}{RGB}{0,51,102}
\definecolor{unifespgreen}{RGB}{0,153,102}

\setbeamercolor{structure}{fg=unifespblue}
\setbeamercolor{palette primary}{bg=unifespblue,fg=white}
\setbeamercolor{palette secondary}{bg=unifespgreen,fg=white}

% Remover navegação
\setbeamertemplate{navigation symbols}{}
\setbeamertemplate{footline}[frame number]

% ============================================================================
% INFORMAÇÕES
% ============================================================================
\title[Agricultura de Precisão e Manejo Hídrico]{%
  Sistema de Agricultura de Precisão para o Manejo Sustentável \\
  de Recursos Hídricos baseado na Fusão de Dados \\
  de Sensores IoT e Imagens de Satélite
}

\author[F. D. Marcelino]{%
  Fernando Daniel Marcelino \\[2mm]
  {\small Orientadora: Profa. Dra. Fernanda Quelho Rossi}
}

\institute[UNIFESP]{%
  Universidade Federal de São Paulo -- UNIFESP \\
  Instituto de Ciência e Tecnologia \\
  Bacharelado em Engenharia de Computação \\
}

\date{8 de Dezembro de 2025}

% Colocando o logo da UNIFESP ao lado direito do título
% \titlegraphic{%
%   \includegraphics[width=2.5cm]{figures/logo-unifesp.pdf}
% }

% ============================================================================
% INÍCIO DO DOCUMENTO
% ============================================================================
\begin{document}

% ----------------------------------------------------------------------------
% SLIDE DE TÍTULO
% ----------------------------------------------------------------------------
\begin{frame}
  \titlepage
\end{frame}

% ----------------------------------------------------------------------------
% SUMÁRIO
% ----------------------------------------------------------------------------
\begin{frame}{Sumário}
  \tableofcontents
\end{frame}

% ============================================================================
% SEÇÃO 1: INTRODUÇÃO
% ============================================================================
\section{Introdução}

\begin{frame}{Contexto e motivação}
  \begin{itemize}
    \item \textbf{Pressão sobre os recursos hídricos:}
      \begin{itemize}
        \item Aumento da \textbf{demanda} global por alimentos
        \item \textbf{Escassez} e \textbf{distribuição desigual} de água
        \item \textbf{Impactos ambientais} da agricultura intensiva
      \end{itemize}
    \item \textbf{Desafio:} produzir mais com \textbf{menor desperdício de água} e \textbf{menor impacto ambiental}
    \item \textbf{Alinhamento com a Agenda 2030 da ONU:}
      \begin{itemize}
        \item ODS 2 -- Fome Zero e Agricultura Sustentável
        \item ODS 6 -- Água Potável e Saneamento
        \item ODS 12 -- Consumo e Produção Responsáveis
      \end{itemize}
  \end{itemize}
\end{frame}

\begin{frame}{Desafios no manejo hídrico}
  \begin{itemize}
    \item Manejo da irrigação muitas vezes baseado em:
      \begin{itemize}
        \item \textbf{Experiência empírica} do produtor
        \item \textbf{Informações dispersas} e \textbf{pouco integradas}
      \end{itemize}
    \item \textbf{Riscos associados:}
      \begin{itemize}
        \item Superirrigação $\Rightarrow$ \textbf{desperdício de água, energia e fertilizantes}
        \item Subirrigação $\Rightarrow$ \textbf{estresse hídrico} e \textbf{perda de produtividade}
      \end{itemize}
    \item Necessidade de \textbf{ferramentas de apoio à decisão} com visão espaço--temporal do uso de água no talhão
  \end{itemize}
\end{frame}

\begin{frame}{Tecnologias habilitadoras}
  \begin{itemize}
    \item \textbf{Internet das Coisas (IoT):}
      \begin{itemize}
        \item Sensores de \textbf{umidade do solo}, \textbf{temperatura/umidade do ar}, \textbf{vazão}, \textbf{pressão}, etc.
        \item Medições de \textbf{alta frequência temporal} diretamente no campo
      \end{itemize}
    \item \textbf{Sensoriamento remoto (SR):}
      \begin{itemize}
        \item \textbf{Imagens multiespectrais} (Sentinel-2, Landsat 8/9, HLS) e \textbf{SAR} (Sentinel-1)
        \item \textbf{Visão} espacialmente \textbf{contínua} da área cultivada
      \end{itemize}
    \item \textbf{Aprendizado de máquina:}
      \begin{itemize}
        \item Modelos para \textbf{estimar indicadores de estado hídrico}
        \item Geração de \textbf{mapas de recomendação de irrigação} em \textbf{nível de talhão}
      \end{itemize}
  \end{itemize}
\end{frame}

% ============================================================================
% SEÇÃO 2: DEFINIÇÃO DO PROBLEMA E OBJETIVOS
% ============================================================================
\section{Definição do Problema e Objetivos}

\begin{frame}{Lacuna na literatura}
  \begin{itemize}
    \item Trabalhos em \textbf{IoT na agricultura}:
      \begin{itemize}
        \item Foco em arquiteturas de \textbf{monitoramento e automação} de irrigação
        \item Ênfase em \textbf{sensores de campo} e lógica de decisão local
      \end{itemize}
    \item Trabalhos em \textbf{sensoriamento remoto e aprendizado de máquina}:
      \begin{itemize}
        \item Estimativa de \textbf{evapotranspiração}, \textbf{umidade do solo}, \textbf{uso de água} e \textbf{zonas de manejo}
      \end{itemize}
    \item Porém:
      \begin{itemize}
        \item Poucas propostas descrevem \textbf{arquiteturas completas} que integrem
              \textbf{séries temporais de sensores IoT e produtos orbitais} em um fluxo operacional
              para \textbf{apoio à decisão em irrigação}
      \end{itemize}
  \end{itemize}
\end{frame}

\begin{frame}{Definição do problema}
  \begin{itemize}
    \item Adoção de IoT e sensoriamento remoto ocorre, em geral, de forma \textbf{fragmentada}
    \item \textbf{Ausência de metodologias} consolidadas de \textbf{fusão de dados} que conciliem:
      \begin{itemize}
        \item \textbf{Alta frequência temporal dos sensores de campo}
        \item \textbf{Cobertura espacial e multiespectral} das imagens de satélite
      \end{itemize}
    
    \begin{alertblock}{Questão Central}
      \textbf{Como integrar, de forma sistemática, dados de sensores IoT e imagens de satélite}
      para \textbf{apoiar decisões de irrigação em agricultura de precisão}, \textbf{reduzindo desperdício de
      água} e \textbf{aumentando a eficiência produtiva}?
    \end{alertblock}
  \end{itemize}
\end{frame}

\begin{frame}{Objetivo geral}
  \begin{block}{Objetivo geral}
    \textbf{Propor e desenvolver} um \textbf{sistema de agricultura de precisão} focado no \textbf{manejo hídrico
    sustentável}, baseado na \textbf{fusão de dados de sensores IoT e imagens de satélite}, com o
    propósito de gerar \textbf{mapas de recomendação de irrigação} mais precisos e eficientes.
  \end{block}
  
%   \vspace{0.4cm}
  
%   \begin{itemize}
%     \item Integração de duas fontes principais de informação:
%       \begin{itemize}
%         \item Medições de redes de sensores IoT em campo
%         \item Dados de sensoriamento remoto orbitais
%       \end{itemize}
%     \item Geração de produtos operacionais em nível de talhão
%   \end{itemize}
\end{frame}

\begin{frame}{Objetivos específicos (1/2)}
  \begin{enumerate}
    \item \textbf{Revisar a literatura} e o estado da arte sobre:
      \begin{itemize}
        \item Tecnologias de \textbf{IoT aplicadas à agricultura}
        \item \textbf{Sensoriamento remoto voltado ao manejo hídrico}
        \item Técnicas de \textbf{fusão de dados e aprendizado de máquina} em \textbf{agricultura de precisão}
      \end{itemize}
    \item \textbf{Definir a arquitetura do sistema}, especificando:
      \begin{itemize}
        \item Componentes de \textbf{\emph{hardware} para coleta de dados}
        \item Plataforma de \textbf{\emph{software} para ingestão e armazenamento}
      \end{itemize}
    \item \textbf{Estabelecer a metodologia para aquisição e processamento de imagens orbitais}
  \end{enumerate}
\end{frame}

\begin{frame}{Objetivos específicos (2/2)}
  \begin{enumerate}
    \setcounter{enumi}{3}
    \item \textbf{Desenvolver um modelo de \emph{machine learning}} para realizar a fusão dos dados,
          \textbf{correlacionando medições pontuais} e de alta frequência (IoT) \textbf{com dados espaciais} e
          de menor frequência (satélite)
    \item Implementar um \textbf{protótipo de sistema} que utilize o modelo de fusão para gerar:
      \begin{itemize}
        \item \textbf{Mapas}
        \item \textbf{Relatórios de irrigação}
        \item \textbf{Dashboards de recomendação e alerta sobre a irrigação}
      \end{itemize}
    \item \textbf{Analisar a eficácia} potencial do sistema na \textbf{redução do desperdício de água} e no
          \textbf{aumento da eficiência produtiva}
  \end{enumerate}
\end{frame}

\begin{frame}{Solução Proposta}
  \begin{center}
    \begin{tikzpicture}[
      node distance=1.0cm,
      box/.style={rectangle, draw, rounded corners, minimum width=1.6cm, minimum height=0.85cm, align=center, font=\scriptsize},
      arrow/.style={-Stealth, thick}
    ]
      % Linha superior: Fontes → ETL → Fusão
      \node[box, fill=blue!20] (iot) {Sensores IoT \\ \tiny (Alta freq.)};
      \node[box, fill=green!20, above=0.5cm of iot] (sat) {Satélites \\ \tiny (Cobertura esp.)};
      
      \node[box, right=1.2cm of iot, yshift=0.25cm, fill=cyan!20] (etl) {ETL \\ \tiny (Limpeza/Transform.)};
      
      \node[box, right=1.2cm of etl, fill=yellow!20] (fusion) {Fusão \\ \tiny (Feature-level)};
      
      % Linha inferior: ML → Produtos
      \node[box, below=1.0cm of fusion, xshift=0.6cm, fill=orange!20] (ml) {ML \\ \tiny (Regressão/Classif.)};
      
      \node[box, right=1.2cm of ml, fill=red!20] (output) {Produtos \\ \tiny (Mapas/Dashboards)};
      
      % Setas
      \draw[arrow] (iot) -- (etl);
      \draw[arrow] (sat) -- (etl);
      \draw[arrow] (etl) -- (fusion);
      \draw[arrow] (fusion) -- (ml);
      \draw[arrow] (ml) -- (output);
    \end{tikzpicture}
  \end{center}
\end{frame}

% ============================================================================
% SEÇÃO 3: MÉTODOS
% ============================================================================
\section{Materiais e Métodos}

\begin{frame}{Visão geral da abordagem}
  \begin{itemize}
    \item Dois caminhos complementares de obtenção de dados:
      \begin{itemize}
        \item \textbf{Caminho A ---} \textbf{uso de \emph{datasets} existentes}
        \item \textbf{Caminho B ---} \textbf{coleta em campo com rede própria baseada em LoRaWAN}
      \end{itemize}
    \item Etapas principais:
      \begin{enumerate}
        \item Gate de decisão (\textbf{Caminho A vs. B})
        \item \textbf{Delineamento experimental}
        \item \textbf{Aquisição de dados} (IoT + SR)
        \item \textbf{Fusão espaço--temporal}
        \item \textbf{Modelagem e geração de produtos}
        \item \textbf{Validação e análise de desempenho}
      \end{enumerate}
  \end{itemize}
\end{frame}

\begin{frame}{Materiais: dados (Caminho A)}
  \begin{itemize}
    \item \textbf{Caminho A --- Dataset existente}
      \begin{itemize}
        \item \textbf{Séries históricas} e/ou \textbf{\emph{datasets}} acadêmicos com \textbf{variáveis de solo/umidade}
              e apoio meteorológico [1][2][3]
        \\{\scriptsize (Bônus: uso de dados meteorológicos de estações locais próximas)}
        \item Sensoriamento remoto: \textbf{Sentinel-2 (MSI), Landsat-8/9 (OLI/TIRS) e Sentinel-1 (SAR)}
        \\{\scriptsize (Bônus: uso de dados vetoriais para auxílio no limite de talhão, rede de drenagem, solos e corpos d'água)}
      \end{itemize}
    \item \textbf{Critérios de escolha:}
      \begin{itemize}
        \item \textbf{Cobertura temporal e espacial}
        \item \textbf{Qualidade das variáveis disponíveis}
      \end{itemize}
  \end{itemize}
\end{frame}

\begin{frame}{Materiais: coleta em campo (Caminho B)}
  \begin{itemize}
    \item \textbf{Caminho B --- Rede própria em 1--2 talhões}
      \begin{itemize}
        \item Medições de \textbf{umidade do solo}
        \item \textbf{Temperatura e umidade relativa do ar}
        \item \textbf{Registros de validação manual} (checklist de campo; amostras pontuais)
      \end{itemize}
    \item \textbf{\emph{Hardware} e comunicação:}
      \begin{itemize}
        \item Microcontrolador \textbf{ESP32 com suporte a LoRa}
        \item \emph{Gateway} LoRaWAN (TTN/ChirpStack) ou Raspberry Pi com módulo LoRa
        \item \textbf{Sensores capacitivos/tensiométricos de umidade, sensores de T/UR do ar}
      \end{itemize}
  \end{itemize}
\end{frame}

\begin{frame}{Materiais: \emph{software} e organização dos dados}
  \begin{itemize}
    \item \textbf{\emph{Software} e serviços:}
      \begin{itemize}
        \item \emph{Ambiente Python} (venv/conda)
        \item \textbf{Bibliotecas:} GDAL/\textbf{rasterio}, SNAP/Sen2Cor, \textbf{Google Earth Engine}
        \item \textbf{Telemetria:} The Things Stack/TTN, \textbf{MQTT/HTTP}, \emph{scripts} de ingestão
        \item Ciência de dados e visualização: \texttt{numpy}, \texttt{pandas}, \texttt{scikit-learn}, QGIS, Dash/Streamlit
      \end{itemize}
    \item \textbf{Organização e gestão de dados:}
      \begin{itemize}
        \item Estrutura de pastas: \texttt{raw/}, \texttt{interim/}, \texttt{processed/}, \texttt{models/}, \texttt{reports/}
        \item \textbf{Metadados} (CSV/Parquet): sensor, lat, lon, data/hora local, DOY, QA \emph{flags}, CRS
        \item Controle de \textbf{versionamento}: \textbf{Git/GitHub}
        \\{\scriptsize (Bônus: o uso do \emph{Data Version Control} (DVC) para grandes volumes de dados)}
      \end{itemize}
  \end{itemize}
\end{frame}

\begin{frame}{\emph{Gate} de decisão e delineamento experimental}
  \begin{itemize}
    \item \textbf{\emph{Gate} de decisão (Mês 1--2):}
      \begin{enumerate}
        \item \textbf{Verificar} disponibilidade e suficiência de \textbf{\emph{datasets} existentes}
        \item \textbf{Se suficientes: adotar Caminho A} para \textbf{acelerar modelagem}
        \item \textbf{Caso contrário: utilizar Caminho B} com \textbf{coleta em campo via LoRaWAN}
      \end{enumerate}
    \item \textbf{Delineamento experimental:}
      \begin{itemize}
        \item \textbf{Definir cultura, talhões e variáveis de interesse}
        \item \textbf{Planejar janelas de aquisição (solo, clima, SR)}
      \end{itemize}
  \end{itemize}
\end{frame}

\begin{frame}{Aquisição de dados e sensoriamento remoto}
  \begin{itemize}
    \item \textbf{Aquisição (IoT + meteorologia):}
      \begin{itemize}
        \item \textbf{Leituras de sensores} em alta frequência
        \item \textbf{Agregação em janelas de 15--60 minutos} para modelagem
      \end{itemize}
    \item \textbf{Sensoriamento remoto (comum aos caminhos A e B):}
      \begin{itemize}
        \item \textbf{Critérios: nuvem $<$ X\%}, \textbf{coleções L2A} (Sentinel-2) / \textbf{C2} (Landsat 8/9)
        \item \textbf{Óptico}: \textbf{correção atmosférica} (Sen2Cor), \textbf{máscara de nuvem/sombra}, \textbf{recorte ao talhão}
        \item \textbf{SAR}: \textbf{calibração radiométrica}, \textbf{correção de terreno}, \textbf{co-registro S1$\leftrightarrow$S2}
        \item \textbf{Derivação de índices}: \textbf{NDWI/MNDWI} (água), NDVI/NDMI (vigor/umidade), LST/ET quando aplicável [4]
      \end{itemize}
  \end{itemize}
\end{frame}

\begin{frame}{Fusão espaço--temporal, modelagem e produtos}
  \begin{itemize}
    \item \textbf{Fusão espaço--temporal:}
      \begin{itemize}
        \item \textbf{Alinhamento temporal/espacial}
        \item \textbf{Uso de \emph{Harmonized Landsat Sentinel-2} (HLS)} e composições para \textbf{reduzir lacunas de nuvem}
        \item Níveis de fusão:
          \begin{itemize}
            \item \textbf{\emph{Sensor-level:}} reamostragem/\textbf{registro} $\leftrightarrow$ pontos/\textbf{talhões}
            \item \textbf{\emph{Feature-level:}} \emph{feature set} com \textbf{índices SR + agregados IoT + clima}
          \end{itemize}
      \end{itemize}
    \item \textbf{Modelagem e produtos:}
      \begin{itemize}
        \item \textbf{Baselines:} \textbf{regressão} (umidade/ET) e \textbf{classificação} (água/estresse)
        \item \textbf{Produtos:} \textbf{GeoTIFF}/COG, camadas temáticas, \textbf{mapas de recomendação de irrigação}, \textbf{dashboards} e \textbf{relatórios de alerta}
      \end{itemize}
  \end{itemize}
\end{frame}

\begin{frame}{Validação}
  \begin{itemize}
    \item \textbf{Esquema espaço--temporal:} \textbf{treino/validação/teste por talhão/tempo}
    \item \textbf{Métricas:} \emph{Root Mean Square Error} (RMSE)/\emph{Mean Absolute Error} (MAE) (variáveis contínuas), F1--\emph{Score}/\emph{Intersection over Union} (IoU) (segmentação água/estresse)
    \item \textbf{Ganho operacional:} \textbf{economia potencial de água}
  \end{itemize}
\end{frame}

\begin{frame}{Cronograma de atividades}
  \begin{columns}[T]
    % Coluna esquerda: Tabela
    \begin{column}{0.48\textwidth}
      \begin{table}
        \centering
        \scriptsize
        \begin{tabular}{cl}
          \toprule
          \textbf{Mês} & \textbf{Atividade} \\
          \midrule
          \textbf{M1--M2} & \textbf{\emph{Gate} de decisão} (A vs. B) \\
                 & Instalação/calibração (se B) \\
          \midrule
          \textbf{M2--M4} & \textbf{Coleta IoT contínua (se B)} \\
                 & Aquisição + pré-proc. SR \\
          \midrule
          \textbf{M4--M5} & \textbf{Fusão de dados} \\
                 & \textbf{Modelagem \emph{baseline}} \\
          \midrule
          \textbf{M5--M6} & \textbf{Validação e ajustes} \\
                 & \textbf{Geração de produtos} \\
                 & \textbf{Escrita Final do TCC} \\
          \bottomrule
        \end{tabular}
      \end{table}
    \end{column}
    
    % Coluna direita: Fluxograma
    \begin{column}{0.48\textwidth}
      \begin{center}
        \begin{tikzpicture}[
          node distance=0.6cm and 0.8cm,
          phase/.style={rectangle, draw, fill=blue!20, rounded corners, minimum width=2cm, minimum height=0.6cm, align=center, font=\tiny},
          decision/.style={diamond, draw, fill=yellow!30, minimum width=1.8cm, minimum height=1.2cm, align=center, font=\tiny, aspect=2},
          arrow/.style={-Stealth, thick},
          label/.style={font=\tiny, midway, above}
        ]
          % M1-M2: Gate de decisão
          \node[phase] (start) {M1--M2 \\ Análise};
          \node[decision, below=0.5cm of start] (gate) {Dataset \\ suficiente?};
          
          % Caminho A (esquerda)
          \node[phase, below left=0.8cm and 0.5cm of gate, fill=green!20] (pathA) {M2--M4 \\ Caminho A \\ (Dataset)};
          
          % Caminho B (direita)
          \node[phase, below right=0.8cm and 0.5cm of gate, fill=blue!30] (pathB) {M2--M4 \\ Caminho B \\ (IoT)};
          
          % Convergência
          \node[phase, below=1.2cm of gate, fill=cyan!20] (merge) {M4--M5 \\ Fusão + ML};
          \node[phase, below=0.5cm of merge, fill=orange!20] (final) {M5--M6 \\ Valid. + TCC};
          
          % Setas
          \draw[arrow] (start) -- (gate);
          \draw[arrow] (gate) -| node[label, pos=0.2] {Sim} (pathA);
          \draw[arrow] (gate) -| node[label, pos=0.2] {Não} (pathB);
          \draw[arrow] (pathA) |- (merge);
          \draw[arrow] (pathB) |- (merge);
          \draw[arrow] (merge) -- (final);
        \end{tikzpicture}
      \end{center}
    \end{column}
  \end{columns}
\end{frame}

% ============================================================================
% SEÇÃO 4: RESULTADOS
% ============================================================================
\section{Resultados}

\begin{frame}{Status Atual do Projeto}
  \begin{table}
    \centering
    \begin{tabular}{lc}
      \toprule
      \textbf{Atividade} & \textbf{Status} \\
      \midrule
      Revisão bibliográfica & \textcolor{green!70!black}{Concluída} \\
      Definição de arquitetura & \textcolor{green!70!black}{Concluída} \\
      Organização do repositório & \textcolor{green!70!black}{Concluída} \\
      Aquisição de dados (A ou B) & \textcolor{orange}{Em decisão} \\
      Pré-processamento de dados & \textcolor{red}{Não iniciada} \\
      Fusão de dados & \textcolor{red}{Não iniciada} \\
      Modelagem & \textcolor{red}{Não iniciada} \\
      Dashboard & \textcolor{red}{Não iniciada} \\
      Validação & \textcolor{red}{Não iniciada} \\
      Escrita da monografia & \textcolor{orange}{Em andamento} \\
      \bottomrule
    \end{tabular}
  \end{table}
\end{frame}

\begin{frame}{Resultados Esperados}
  \begin{enumerate}
    \item \textbf{Metodologia de Fusão}
      \begin{itemize}
        \item Pipeline reproduzível e documentado
        \item Integração efetiva de IoT + Satélite
      \end{itemize}
    
    \item \textbf{Modelo Preditivo}
      \begin{itemize}
        \item Estimativa de umidade do solo (R² > 0.8)
        \item Classificação de necessidade de irrigação (F1 > 0.75)
      \end{itemize}
    
    \item \textbf{Produtos Operacionais}
      \begin{itemize}
        \item Mapas de recomendação de irrigação
        \item Dashboard de apoio à decisão
        \item Relatórios de economia de água
      \end{itemize}
    
    \item \textbf{Impacto}
      \begin{itemize}
        \item Redução estimada de 20-30\% no uso de água
        \item Melhoria na uniformidade da irrigação
        \item Contribuição para ODS 2, 6 e 12
      \end{itemize}
  \end{enumerate}
\end{frame}

\begin{frame}{Exemplo de Produto Final}
  \begin{center}
    \includegraphics[width=0.7\textwidth]{figures/example_irrigation_map.png}
    \\[3mm]
    {\small Exemplo conceitual de mapa de recomendação de irrigação}
  \end{center}
  
  \begin{itemize}
    \item \textcolor{blue}{Azul}: Não irrigar (umidade adequada)
    \item \textcolor{yellow}{Amarelo}: Irrigação moderada
    \item \textcolor{red}{Vermelho}: Irrigação intensa (déficit hídrico)
  \end{itemize}
\end{frame}

% ============================================================================
% SEÇÃO 5: CONCLUSÕES E TRABALHOS FUTUROS
% ============================================================================
\section{Conclusões e Trabalhos Futuros}

\begin{frame}{Conclusões Parciais}
  \begin{itemize}
    \item \textbf{Revisão bibliográfica} confirma:
      \begin{itemize}
        \item Viabilidade técnica da fusão IoT + Satélite
        \item Lacuna na integração operacional dessas fontes
        \item Potencial de ML para agricultura de precisão
      \end{itemize}
    
    \vspace{0.3cm}
    
    \item \textbf{Arquitetura proposta} é:
      \begin{itemize}
        \item Modular e escalável
        \item Baseada em tecnologias abertas
        \item Adaptável a diferentes cenários (Caminhos A e B)
      \end{itemize}
    
    \vspace{0.3cm}
    
    \item \textbf{Metodologia} está:
      \begin{itemize}
        \item Bem fundamentada teoricamente
        \item Alinhada com boas práticas de ciência de dados
        \item Pronta para implementação
      \end{itemize}
  \end{itemize}
\end{frame}

\begin{frame}{Trabalhos Futuros}
  \begin{columns}
    \column{0.5\textwidth}
    \textbf{Curto Prazo (TCC):}
    \begin{enumerate}
      \item Decisão sobre fonte de dados
      \item Implementação do pipeline
      \item Treinamento de modelos
      \item Desenvolvimento do dashboard
      \item Validação dos resultados
      \item Finalização da monografia
    \end{enumerate}
    
    \column{0.5\textwidth}
    \textbf{Longo Prazo:}
    \begin{enumerate}
      \item Expansão para outras culturas
      \item Modelos de previsão (LSTM)
      \item Integração com sistemas de automação
      \item App mobile para agricultores
      \item Análise de custo-benefício
      \item Estudo de caso em propriedades reais
      \item Publicação científica
    \end{enumerate}
  \end{columns}
\end{frame}

\begin{frame}{Contribuições Esperadas}
  \begin{block}{Contribuições Técnicas}
    \begin{itemize}
      \item Metodologia de fusão IoT-Satélite documentada
      \item Código e arquitetura open-source
      \item Pipeline reproduzível de dados
    \end{itemize}
  \end{block}
  
  \begin{block}{Contribuições Científicas}
    \begin{itemize}
      \item Validação de abordagem integradora
      \item Quantificação de benefícios da fusão
      \item Base para estudos futuros
    \end{itemize}
  \end{block}
  
  \begin{block}{Contribuições Socioambientais}
    \begin{itemize}
      \item Redução do desperdício de água
      \item Apoio à agricultura sustentável
      \item Alinhamento com ODS da ONU
    \end{itemize}
  \end{block}
\end{frame}

% ============================================================================
% SLIDE FINAL
% ============================================================================
\begin{frame}[plain]
  \begin{center}
    {\Huge Obrigado!}
    
    \vspace{1cm}
    
    {\Large Dúvidas e Sugestões?}
    
    \vspace{1.5cm}
    
    \begin{columns}
      \column{0.5\textwidth}
      \begin{center}
        \textbf{Fernando Daniel Marcelino} \\[2mm]
        \href{mailto:fernando.marcelino@unifesp.br}{fernando.marcelino@unifesp.br} \\[5mm]
        \includegraphics[width=0.3\textwidth]{figures/logo-unifesp.pdf}
      \end{center}
      
      \column{0.5\textwidth}
      \begin{center}
        \textbf{Repositório do Projeto:} \\[2mm]
        \href{https://github.com/fernando-daniel98/TCC-UNIFESP}{github.com/fernando-daniel98/TCC-UNIFESP} \\[5mm]
        {\small Código, dados e documentação disponíveis}
      \end{center}
    \end{columns}
  \end{center}
\end{frame}

\end{document}
